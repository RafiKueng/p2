\documentclass[fleqn,usenatbib]{template/mnras}

\pdfoutput=1
% pdfoutput is used for arxiv, should be somewhere in lines 1-5




% Use vector fonts, so it zooms properly in on-screen viewing software
% Don't change these lines unless you know what you are doing
\usepackage[T1]{fontenc}      % make sure font is supported
\usepackage[final]{microtype} % make sure font is supported
\usepackage{lmodern}          % use a modern font with T1 support

% Only include extra packages if you really need them. Common packages are:
\usepackage{graphicx} % Including figure files
\usepackage{amsmath}  % Advanced maths commands
\usepackage{amssymb}  % Extra maths symbols
\usepackage{pifont}   % Extra maths symbols
\usepackage{xspace}   % handling of spaces after shortcuts
\usepackage{siunitx}


%%% DEBUG SETTINGS - remove at the end %%%%%%%%%%%%%%%
% \usepackage{todonotes} % as long as we are editing...
% \usepackage{showframe} % for debug
% \overfullrule=5pt      % show overfull boxes

% I disabled hyperlink in the cls und included it here because of bugs
\usepackage{hyperref}   % Hyperlinks
\hypersetup{
  colorlinks=true,
  linkcolor=blue,
  citecolor=blue,
  filecolor=blue,
  urlcolor=blue,
  pdfauthor={Kueng, Rafael et al.},
  pdftitle={Models of gravitational lens candidates from Space Warps CFHTLS}
}

\newcommand*{\rot}{\rotatebox{90}}
\newcommand*{\OK}{\ding{51}}
\newcommand*{\NO}{\ding{55}}
\newcommand*{\UK}{\ensuremath{\text{--}}}

\newcommand{\SW}{Space\,Warps\xspace}
\newcommand{\SpL}{SpaghettiLens\xspace}
\newcommand{\asw}[1]{ASW000\,#1\xspace}
\newcommand{\sw}[1]{SW\,#1\xspace}
%\newcommand{\model}[1]{SL model~#1}

\newcommand{\figref}[1]{\ref{fig:#1}}

\newcommand{\Mstel}{\ensuremath{M_\text{stel}}}
\newcommand{\Mhalo}{\ensuremath{M_\text{halo}}}
\newcommand{\Msun}{\ensuremath{M_\odot}}
\newcommand{\Mone}{\ensuremath{M_1}}
\newcommand{\haloindex}{\mathcal{H}}
%\newcommand{\ER}{$\Theta_{\text{E}}$} % einstein radius
\newcommand{\ER}{\ensuremath{r_\text{E}}\xspace} % einstein radius

\newcommand{\sqdeg}{\ensuremath{\,\text{deg}^2}\xspace}
\renewcommand{\arcsec}{\ensuremath{\text{arcsec}}\xspace}

\def\pwidth{.32\linewidth}



% Look up table, SWID -> Index:
% 02 AC
% 04 AE
% 05 AF
% 06 AG
% 08 AI
% 09 AJ
% 10 BA
% 11 BB
% 12 BC
% 13 BD
% 15 BF
% 16 BG
% 17 BH
% 18 BI
% 19 BJ
% 20 CA
% 21 CB
% 22 CC
% 23 CD
% 24 CE
% 26 CG
% 27 CH
% 28 CI
% 29 CJ
% 31 DB
% 33 DD
% 34 DE
% 35 DF
% 36 DG
% 38 DI
% 41 EB
% 42 EC
% 43 ED
% 44 EE
% 45 EF
% 46 EG
% 47 EH
% 53 FD
% 54 FE
% 56 FG
% 57 FH
% 58 FI

\newcommand{\inclAC}[1]{\includegraphics[width=\pwidth]{img/#1/SW02_ASW000619d_011489_#1}}
\newcommand{\inclAE}[1]{\includegraphics[width=\pwidth]{img/#1/SW04_ASW0009cjs_NJ5CC5YJAQ_#1}}
\newcommand{\inclAF}[1]{\includegraphics[width=\pwidth]{img/#1/SW05_ASW0007k4r_012771_#1}}
\newcommand{\inclAG}[1]{\includegraphics[width=\pwidth]{img/#1/SW06_ASW0008swn_BCY2NOUSLK_#1}}
\newcommand{\inclAI}[1]{\includegraphics[width=\pwidth]{img/#1/SW08_ASW00099ed_HISGRAIZL2_#1}}
\newcommand{\inclAJ}[1]{\includegraphics[width=\pwidth]{img/#1/SW09_ASW0002asp_5EKMWWVJHL_#1}}
\newcommand{\inclBA}[1]{\includegraphics[width=\pwidth]{img/#1/SW10_ASW0002bmc_VQYCYNONVW_#1}}
\newcommand{\inclBB}[1]{\includegraphics[width=\pwidth]{img/#1/SW11_ASW0002qtn_3TUJKHGED4_#1}}
\newcommand{\inclBC}[1]{\includegraphics[width=\pwidth]{img/#1/SW12_ASW0003wsu_012712_#1}}
\newcommand{\inclBD}[1]{\includegraphics[width=\pwidth]{img/#1/SW13_ASW00047ae_TGTIIF7HCV_#1}}
\newcommand{\inclBF}[1]{\includegraphics[width=\pwidth]{img/#1/SW15_ASW0004nan_QUOGDU2NN6_#1}}
\newcommand{\inclBG}[1]{\includegraphics[width=\pwidth]{img/#1/SW16_ASW0009bp2_013421_#1}}
\newcommand{\inclBH}[1]{\includegraphics[width=\pwidth]{img/#1/SW17_ASW0005rnb_AAKHMTYTMS_#1}}
\newcommand{\inclBI}[1]{\includegraphics[width=\pwidth]{img/#1/SW18_ASW0007hu2_D4UQI6M3ZU_#1}}
\newcommand{\inclBJ}[1]{\includegraphics[width=\pwidth]{img/#1/SW19_ASW0001ld7_OS3CYAKLRT_#1}}
\newcommand{\inclCA}[1]{\includegraphics[width=\pwidth]{img/#1/SW20_ASW0002dx7_3NYJG67KRT_#1}}
\newcommand{\inclCB}[1]{\includegraphics[width=\pwidth]{img/#1/SW21_ASW0004m3x_QZROE23AUH_#1}}
\newcommand{\inclCC}[1]{\includegraphics[width=\pwidth]{img/#1/SW22_ASW0009ab8_TGM4U2TZBS_#1}}
\newcommand{\inclCD}[1]{\includegraphics[width=\pwidth]{img/#1/SW23_ASW0003r61_002481_#1}}
\newcommand{\inclCE}[1]{\includegraphics[width=\pwidth]{img/#1/SW24_ASW00050sk_013406_#1}}
\newcommand{\inclCG}[1]{\includegraphics[width=\pwidth]{img/#1/SW26_ASW0005ma2_5ZZKUM3SWL_#1}}
\newcommand{\inclCH}[1]{\includegraphics[width=\pwidth]{img/#1/SW27_ASW0006jh5_5URN3BQFSV_#1}}
\newcommand{\inclCI}[1]{\includegraphics[width=\pwidth]{img/#1/SW28_ASW0007xrs_JHC3J2HYV7_#1}}
\newcommand{\inclCJ}[1]{\includegraphics[width=\pwidth]{img/#1/SW29_ASW0008qsm_TOFS7JNGEK_#1}}
\newcommand{\inclDB}[1]{\includegraphics[width=\pwidth]{img/#1/SW31_ASW00021r0_CIFSR5JFFU_#1}}
\newcommand{\inclDD}[1]{\includegraphics[width=\pwidth]{img/#1/SW33_ASW0003s0m_ECXCIRBDUJ_#1}}
\newcommand{\inclDE}[1]{\includegraphics[width=\pwidth]{img/#1/SW34_ASW00051ld_000291_#1}}
\newcommand{\inclDF}[1]{\includegraphics[width=\pwidth]{img/#1/SW35_ASW0004wgd_VWJ2LNN3VZ_#1}}
\newcommand{\inclDG}[1]{\includegraphics[width=\pwidth]{img/#1/SW36_ASW000096t_7IPP7LWVOF_#1}}
\newcommand{\inclDI}[1]{\includegraphics[width=\pwidth]{img/#1/SW38_ASW0009cp0_Z6IFI4SLLM_#1}}
\newcommand{\inclEB}[1]{\includegraphics[width=\pwidth]{img/#1/SW41_ASW0008xbu_BFXRMIQEAT_#1}}
\newcommand{\inclEC}[1]{\includegraphics[width=\pwidth]{img/#1/SW42_ASW00096rm_4Q3YCEWGLN_#1}}
\newcommand{\inclED}[1]{\includegraphics[width=\pwidth]{img/#1/SW43_ASW0001c3j_5R6UYQZUTI_#1}}
\newcommand{\inclEE}[1]{\includegraphics[width=\pwidth]{img/#1/SW44_ASW0002k40_000899_#1}}
\newcommand{\inclEF}[1]{\includegraphics[width=\pwidth]{img/#1/SW45_ASW00024id_013557_#1}}
\newcommand{\inclEG}[1]{\includegraphics[width=\pwidth]{img/#1/SW46_ASW00024q6_012523_#1}}
\newcommand{\inclEH}[1]{\includegraphics[width=\pwidth]{img/#1/SW47_ASW0003r6c_4HC3CREEAD_#1}}
\newcommand{\inclFD}[1]{\includegraphics[width=\pwidth]{img/#1/SW53_ASW00070vl_BPAV4GVOPP_#1}}
\newcommand{\inclFE}[1]{\includegraphics[width=\pwidth]{img/#1/SW54_ASW0007sez_SI4ELBAKL2_#1}}
\newcommand{\inclFG}[1]{\includegraphics[width=\pwidth]{img/#1/SW56_ASW0007pga_VHV6RQYYKZ_#1}}
\newcommand{\inclFH}[1]{\includegraphics[width=\pwidth]{img/#1/SW57_ASW0008pag_5SXGXQYY6V_#1}}
\newcommand{\inclFI}[1]{\includegraphics[width=\pwidth]{img/#1/SW58_ASW0007iwp_4XBJWT3COV_#1}}

\newcommand{\inclGrid}[1]{ %
\inclFI{#1} \inclCI{#1} \inclFH{#1}
\inclAF{#1} \inclEC{#1} \inclBJ{#1}
\inclAJ{#1} \inclCJ{#1} \inclAC{#1}
}




\title[Lens models for Space Warps CFHTLS]{Models of gravitational
    lens candidates from Space Warps CFHTLS}

\author[K\"ung et al]{Rafael K\"ung,$^{1}$
Prasenjit Saha,$^{1}$
Ignacio Ferreras,$^{2}$
Elisabeth Baeten,$^{3}$
\newauthor
Jonathan Coles,$^{4}$
Claude Cornen,$^{3}$
Christine Macmillan,$^{3}$
Phil Marshall,$^{5}$ 
\newauthor
Anupreeta More,$^{6}$
Lucy Oswald$^{7}$
Aprajita Verma$^{8}$
and Julianne K. Wilcox$^{3}$
%
\\
%
$^{1}$Physik-Institut, University of Zurich, Winterthurerstrasse 190, 8057 Zurich, Switzerland\\
$^{2}$Mullard Space Science Laboratory, University College London, Holmbury St Mary, Dorking, Surrey RH5 6NT, UK\\
$^{3}$Zooniverse, c/o Astrophysics Department, University of Oxford, Oxford OX1 3RH, UK \\
%$^{4}$Exascale Research Computing Lab, Campus Teratec, 2 Rue de la Piquetterie, 91680 Bruyeres-le-Chatel, France\\
$^{4}$Physik-Department, Technische Universit\"at M\"unchen
James-Franck-Str.~1, 85748 Garching, Germany\\
$^{5}$Kavli Institute for Particle Astrophysics and Cosmology, Stanford University, 452 Lomita Mall, Stanford, CA 94035, USA\\
%$^{6}$Kavli Institute for the Physics and Mathematics of the Universe, University of Tokyo, 5-1-5 Kashiwanoha, Kashiwa-shi 277-8583, Japan\\
$^{6}$Kavli IPMU (WPI), UTIAS, University of Tokyo, Kashiwa, Chiba 277-8583, Japan\\
$^{7}$Murray Edwards College, University of Cambridge, Cambridge CB3 0DF, UK\\
$^{8}$Sub-department of Astrophysics, University of Oxford, Denys Wilkinson Building, Keble Road, Oxford, OX1 3RH, UK\\
}



% These dates will be filled out by the publisher
\date{Accepted 2017 November 19. Received 2017 November 19; in original form 2017 May 02}

% Enter the current year, for the copyright statements etc.
\pubyear{2017}

% Don't change these lines
\begin{document}
\label{firstpage}
\pagerange{\pageref{firstpage}--\pageref{lastpage}}
\maketitle

\begin{abstract}
We report modelling follow-up of recently discovered
gravitational-lens candidates in the Canada France Hawaii Telescope
Legacy Survey. Lens modelling was done by a small group of
specially interested volunteers from the \SW citizen-science community
who originally found the candidate lenses.  Models are categorized
according to seven diagnostics indicating (a)~the image morphology and
how clear or indistinct it is, (b)~whether the mass map and synthetic
lensed image appear to be plausible, and (c)~how the lens-model mass
compares with the stellar mass and the abundance-matched halo mass.
The lensing masses range from $\sim10^{11}M_\odot$ to
$>10^{13}M_\odot$. Preliminary estimates of the stellar masses show a
smaller spread in stellar mass (except for two lenses): a factor of a
few below or above $\sim10^{11}M_\odot$.  Therefore, we expect the
stellar-to-total mass fraction to decline sharply as lensing mass
increases.  The most massive system with a convincing model is
J1434+522 (\sw{05}).  The two low-mass outliers are J0206$-$095
(\sw{19}) and J2217+015 (\sw{42}); if these two are indeed lenses,
they probe an interesting regime of very low star formation
efficiency.  Some improvements to the modelling software
(SpaghettiLens), and discussion of strategies regarding scaling to
future surveys with more and frequent discoveries, are included.
\end{abstract}

\begin{keywords}
gravitational lensing: strong -- galaxies: general
-- galaxies: stellar content -- dark matter.
\end{keywords}

\section{Introduction}

By coincidence, the typical escape velocity of massive galaxies is
such that $v_{\rm esc}^2/c^2$ is comparable to the apparent sizes of a
galaxies at cosmological distances.  This coincidence is fortunate,
because it makes the lensing deflection angle (which is $2v_{\rm
  esc}^2/c^2$) of distant galaxies comparable to their size on the
sky, and as a result, strong lensing by galaxies tends to produce
images that probe the dark halos of those galaxies.  This is
important, because while there is a general consensus that basic
mechanism of galaxy formation involve gravitational collapse,
fragmentation, and mergers of dark-matter clumps, into which gas fell,
cooling through radiative processes to form dense clouds and
eventually stars, there is much debate about the details \citep[for a
  summary, see][]{2012RAA....12..917S}.  In particular, the nature of
dark matter remains mysterious: most researchers take it to be a
collisionless non-relativistic fluid (cold dark matter or CDM) readily
studied by simulations \citep[for example, the influential millenium
  simulation by][]{2005Natur.435..629S}.  But other scenarios have
also been considered, such as a cold condensing boson fluid
\citep{2016ApJ...818...89S}, or dark matter particles with internal
degrees of freedom \citep{2010MNRAS.405...77S}, or not matter at all
but a modification of gravity \citep{2016PhRvL.117t1101M}.

All this motivates using galaxy lenses to study the mutual dynamics of
dark matter and gas in galaxies.  Several studies in recent years have
done so
\citep{2009ApJ...703L..51K,2011ApJ...740...97L,2012MNRAS.424..104L,
  2016MNRAS.459.3677L,2016MNRAS.456..870B} but it is desirable to
enlarge the samples from tens of lensing galaxies to thousands.  Doing
so requires both finding more lenses and also modelling their masses.
Recent searches through the CFHTLS \citep{2012SPIE.8448E..0MC} using
arc-finders
\citep{2012ApJ...749...38M,2014A&A...567A.111M,2014ApJ...785..144G} by
machine learning \citep{2016A&A...592A..75P} and by visual inspection
by citizen-science volunteers in Space~Warps
\citep{2016MNRAS.455.1191M} have, between them, discovered an average
of four lenses per square degree, so one can be optimistic about
finding many thousands of lenses in the next generation of wide-field
surveys, from the LSST in optical and the SKA in radio on the ground,
and Euclid and WFIRST in orbit.

The expected flood of new lens discoveries will need a similarly huge
modelling effort to reconstruct their mass distributions.  To prepare
for the challenge of massive-sample lens modelling,
\cite{2015MNRAS.447.2170K} developed a new modelling strategy,
implemented as the SpaghettiLens system.  The idea is to collaborate
with experienced members of the citizen-science community, who have
already participated in lens discovery through Space~Warps, as well as
several other projects involving astronomical data.  The system was
tested on a sample of simulated lenses from Space~Warps.

This paper continues by applying SpaghettiLens to lens-candidates
discovered through Space~Warps.  We present results from modelling of
56 of the 59 lens candidates reported by \cite{2016MNRAS.455.1191M}.
Each lens candidate was modelled in a collaborative refinement
process, where anyone interested can create a new model or modify an
existing model to try and make it better.\footnote{This is in contrast
  to the main Space~Warps project for discovering lenses, where
  volunteers in a crowd of $\gtrsim10^4$ make independent
  contributions.  Each person is presented with a random selection of
  survey-patches and invited to (in effect) vote on each.  The system
  estimates each volunteer's skill level according to test-patches
  interspersed with the real data, and weights their votes accordingly
  \citep{2016MNRAS.455.1171M}.  There is an active forum for
  volunteers, but since everyone is seeing different data samples with
  minimal overlap, the forum has little if any influence on votes.}
The result model represents a consensus among contributors, as to the
best that could be achieved with the available data and software.

We characterise each model with seven diagnostics, grouped into three 
categories, whose purpose is to help identify which systems are most probably 
lenses, and which ones are likely to be most rewarding for future follow-up 
observations. The diagnostics are as follows.

\begin{itemize}
\item First we have diagnostics based on morphology of the
  system.
  Section~\ref{sec:morph} and Figure~\ref{fig:splinput} explain.
\begin{itemize}
\item Whether the images are unblended.  Distinct unblended images are
  an advantage in modelling, but not essential.
\item Whether all images are discernible.  The topography of an
  arrival-time surface, as encoded by a spaghetti diagram, may require
  more images than are visible, in which case the modeller has to
  insert conjectural image positions.
\item Whether the lens is fairly isolated.
\item The image morphology concisely described: double or quads,
  further sub-categorised to indicate the elongation direction of the
  lensing mass.
\end{itemize}
\item Second we have mass models, covered in Section~\ref{sec:massmodels}.
\begin{itemize}
\item Whether the mass map is reasonable. Figure~\ref{fig:kappa}.
\item Whether the arrival-time surface and synthetic image are
  plausible.  In particular, additional images are implied in regions
  where they are not observed signal an unsatisfactory model.
  Figures~\ref{fig:arriv} and \ref{fig:synth} and \ref{fig:encl}.
\end{itemize}
\item Third, whether the implied lensing mass is plausible, given the
  photometric data of the lensing galaxy.  Section~\ref{sec:stellar-mass}
  explains how we compare the lensing mass with the mass in stars in
  the lensing galaxy.  We estimate the stellar mass by comparing
  galaxy magnitudes from the CFHTLS pipeline with the well-known
  stellar-population models of \cite{2003MNRAS.344.1000B}.  We then
  extrapolate the stellar mass to a halo mass using the
  abundance-matching prescription of \cite{2010ApJ...710..903M}.
  Naturally, the lensing mass must be more than the stellar mass but
  no more than the total halo mass.  We then introduce what we call a
  halo index ($\haloindex$) which gives an idea of how the lensing mass
  compares with these two contraints.  Figure~\ref{fig:stelmass}.
\end{itemize}

Section~\ref{sec:summary} summarises and tabulates the diagnostics in
Table~\ref{tab:models}.  Interpretation of the results is preliminary,
because the systems are candidates at this stage, not secure lenses.
Moreover the candidate-lens redshifts have large uncertainties, while
the candidate-source redshifts can only be guessed at present.
Nevertheless it is interesting to see what trends we can observe with
the already-available data.

There are three appendices devoted to various technical issues
relating to modelling.  \cite{2015MNRAS.447.2170K} tested the system
on simulated lenses and identified some areas for improvement.  In
\S~\ref{subsec:sourcefit} we introduce fitting of the brightness
profiles of the source.  This feature has not yet been included in
SpaghettiLens, but has been carried out in post-processing for a few
especially interesting candidates.  In \S~\ref{subsec:hires} we show
that making mass maps fine-grained in the central region relieves a
tendency in the earlier work for mass to be too shallow. Then in
\S~\ref{subsec:parameter} we consider the possibility of fitting a
parametric lens model to the model ensemble; so far we have only been
successful at extracting an Einstein radius.

The online supplement gives results for all the modelled systems generated
for all the lensing candidates.



\section{Image morphology}

Figure~\ref{fig:markedup} shows cutouts of the Space~Warps image,
marked up with a spaghetti diagram.

The lenses are arranged as follows.
$$
\begin{matrix}
\rm SW05 & \rm SW42 & \rm SW28 \\
\rm SW58 & \rm SW02 & \rm SW19 \\
\rm SW09 & \rm SW29 & \rm SW57
\end{matrix}
$$ 
$\{${\em Panels will be labelled, and this little table removed.}$\}$

We show modelling results for nine of the lens candidates.  The first
five (see Figures~\figref{SW05}--\figref{SW29}) are the most
convincingly modelled systems.  The next three cases
(Figures~\figref{SW42}--\figref{SW19}) are less good but still very
plausible models, and are representative of the majority of the
sample.  The last two are examples where the models were unconvincing
(Figures~\figref{SW36}) or completely failed (\figref{SW57}).


Let us first consider from SW05 (J143454.4+522850).
In SW05 there are four distinct lensed images, and the spaghetti
diagram proposes that they are a minimum nearest to the lensing
galaxy, a close minimum-saddle pair, and a minimum further away. In
Table~\ref{tab:models} we refer to such configurations as IQ for
inclined quad.

Figure~\figref{SW42} shows SW42.  The image morphology is similar to
SW05 in Figure~\figref{SW05}, but the lens is much smaller on the sky.

Proceeding to Figure~\figref{SW28}, we have one image close to the
galaxy and an arc further away, which is interpreted as a blend of
three images (a saddle point with two minima on either side).  We call
this a long-axis quad or LQ configuration.  It is an indication of a
mass distribution elongated along the arc-counterimage direction
(along EW in this case).

In Figures~\figref{SW02}--\figref{SW29} we see three candidates with
an arc close the lensing galaxy and one image further away.  The arc
is interpreted three images (a minimum with two saddle points on
either side).  We call this a short-axis quad or SQ configuration.  It
is an indication of a mass distribution elongated perpendicular to the
arc-counterimage direction.



\begin{figure*}
\includeten{spaghetti/}{_input}
\caption{Marked-up images. \label{fig:markedup}}
\end{figure*}



\section{Mass models}\label{sec:massmodels}

Once a spaghetti diagram has been drawn on a web browser, it is
forwarded to a server-side numerical framework, which searches for
mass maps consistent with the given image locations, parities and time
ordering.  The mass maps are made up of mass tiles and are free-form,
but are required to be concentrated around the identified lens centre
\citep[see][for the precise formulation of the search
  problem.]{2014MNRAS.445.2181C} Assuming such mass distributions can
be found (in practise, usually the case) a statistical ensemble of
two-dimensional mass maps is returned.  This ensemble, along with
derived quantities and uncertainties, makes up one SpaghettiLens
model.

The mass distribution will naturally depend on the lens and source
redshifts, which are unknown when a lens candidate is first
identified.  But this is not a problem, because a model can be
trivially rescaled to use better redshift values, as and when they
become available.  This paper uses pipeline photometric redshifts for
the candidate lensing galaxies.  Source redshifts are set to $z=2$
unless an unambiguous photometric redshift is available.

Figure~\ref{fig:arriv}

Figure~\ref{fig:synth}
New method, see also Figure~\figref{synthimg} in
Appendix~\ref{subsec:sourcefit}.

Figure~\ref{fig:kappa}

Figure~\ref{fig:encl}. Appendix~\ref{subsec:hires} describes
improvements made since our earlier work \citep{2015MNRAS.447.2170K}



\endinput



  The ensemble average
can, however, be a useful representative of the whole ensemble.  The
user interface of SpaghettiLens returns graphical representations of
the ensemble-average mass map and some derived quantities to the
modeller for review.  The modeller can post these results on a forum,
or discard them and try again.  Other volunteers can start the
modelling process afresh, or they can take an existing model from the
forum and modify its input spaghetti diagram or its accompanying
options, and thus obtain a revised model.



Alongside is a synthetic image produced by modelling.  In it, the
fitted lensed images are shown in colour, while the lensing galaxy and
extraneous objects have been reduced to grayscale.
) These two
panels are qualitative and display no units, and moreover, the mutual
alignment of the two panels is only approximate.

Figures~\figref{SW58} and \figref{SW19} appear plausible lens
candidates.  Their morphology is similar to SW28, but the saddle-point
counterimage is not visible.  We consider these lenses plausible but
less convincing.

Figures~\figref{SW36} and \figref{SW57} are cases where modelling
failed.

\begin{enumerate}
\item The middle row shows contour maps from the model.  At middle
  left, we have the arrival-time surface.
\item At middle right we have the mass distribution in the usual
  dimensionless form $\kappa$.  Again, these two panels are mainly
  qualitative: both panels are spatially registered and centred on the
  density-peak of the lens, but no scales have been included.
\item The base row supplies spatial and mass scales.  The left panel

\item 
\end{enumerate}



\section{Stellar and halo mass estimates}\label{sec:stellar-mass}

The stellar masses of the lens galaxies are derived by comparison of
the photometric data with M/L estimates from population synthesis
models.  In principle, a detailed analysis of the spectral energy
distribution is needed to derive accurate stellar masses
\citep[e.g.][]{2009ApJS..185..253G,2011MNRAS.418.1587T}.  However,
estimates to within 0.3\,dex in $\log($M$_s$/M$_\odot)$ can be derived
with a single colour, preferably tracing a rest-frame colour similar
to $U-V$ \citep[see fig.~1 of][]{2008MNRAS.383..857F}. 

In this paper we further simplify the analysis by assuming a
relationship between the apparent total magnitude and stellar mass, at
the redshift of the lens.  For typical stellar population parameters,
the variation of this relation is at most 1\,dex.  A further possible
systematic error is contamination of the light of the lensing galaxy
by the lensed galaxy.  Reducing or eliminating the latter would
require detailed fitting of light distributions for each candidate,
which we have not yet attempted.  These caveats notwithstanding, it is
still interesting to compare the derived stellar mass with the lensing
mass.

We make use of the \citet{2003MNRAS.344.1000B} models to derive two
functional forms of the stellar mass with respect to SDSS-$i$ band
magnitudes. The models have solar metallicity, with a Chabrier IMF,
and assume two different age trends: a ``young'' model, with a
constant 500\,Myr age at all redshifts, and an ``old'' model where the
age is the oldest one possible at each redshift, adopting a standard
$\Lambda$CDM model with $H_0=70$\,km\,s$^{-1}$\,Mpc$^{-1}$ and
$\Omega_m=0.3$. We take a geometric mean of the output from these two
models as the stellar mass of the lens. This rough estimate will be
improved in future versions by use of the available optical and NIR
magnitudes to derive more accurate constrains on the stellar
populations.

In addition, we also derive halo masses for the lenses by use of the
standard abundance matching technique, whereby a comparison of the
observed stellar mass function of galaxies with the dark matter halo
mass function from N-body simulations results in a simple relation
between the two. We emphasize that this derivation of halo mass should
be considered an ``average'' estimate, and a significant scatter can
be expected as galaxies with the same stellar mass can be found in
different environments. We refer the reader to \cite{2012MNRAS.424..104L}
for an assessment of the effect of abundance matching on the
derivation of dark matter halo properties in lensing galaxies. We
follow the prescription of \citet{2010ApJ...710..903M}, namely:

\begin{equation}
\begin{aligned}
\frac{\Mstel}{\Mhalo} &= \frac{2C_0}{(\Mhalo/M_1)^{-\beta} +
                                     (\Mhalo/M_1)^\gamma} \\
C_0 &= 0.02820, \quad M_1 = 10^{11.884} M_\odot \\
\beta &= 1.057, \quad \gamma = 0.556.
\end{aligned}
\end{equation}
and define a halo-matching index:
\begin{equation}
\haloindex = \frac{\ln(M/\Mstel)}{\ln(\Mhalo/\Mstel)}
\end{equation}
that relates the observed lensing to stellar mass, with the
global ratio expected if the host halo corresponds to the
average value derived by abundance matching. Several cases
for $\haloindex$ can be considered:
\begin{itemize}
\item $\haloindex < 0$ is unphysical because $M<\Mstel$.
\item $\haloindex = 0$ is when the stellar mass exactly accounts for the
  lensing mass.
\item $0 < \haloindex < 1$ is the typical situation, where the lens
  includes stars and dark matter, but not the full halo.
\item $\haloindex = 1$ means that the lens consists of the entire halo.
\item $\haloindex > 1$ is in tension with abundance-matching, because the
  lensing mass exceeds the expected halo mass.
\end{itemize}



\section{Summary of models}

Now we give a concise characterisation of each modelled system,
according to the image morphology and how clear or indistinct it is,
whether the mass map and synthetic lensed image appear to be
plausible, and how the model mass compares with the estimated stellar
and full-halo masses.

Table~\ref{tab:models} shows a summary of the analysis for many candidates.
Missing rows are due to photometry data not available.

SW05, SW28, SW02, SW09 and SW29 all correspond to the mass range of
massive ellipticals.  SW05 is the most massive of all the candidates,
with a galaxy-group scale mass.

For SW42 the inferred mass is at the low end of the sample.  (The
synthetic image shown is cruder than for the five previous figures,
but that does not influence the mass models.)  If these preliminary
findings are confirmed by follow-up observations, this could be the
most dark-matter dominated lens known.\todo{Check photometry of this
  candidate.}




\bibliographystyle{template/mnras}
\bibliography{bib/bibli}

\clearpage

\begin{figure*}
\inclGrid{spl-input}
\caption{Nine of the lens candidates marked up with spaghetti
  diagrams.  Red, blue and green dots are proposed locations for
  maxima, minima and saddle points of the arrival time.  The curves
  help guide the placement of the dots, but their precise appearance
  has no significance.  This selection includes the best-modelled
  systems, but also one case (SW57 at upper right) of unsuccessful
  modelling.  Since the modelling process is collaborative among the
  volunteers, with anyone welcome to contribute new models or modify
  existing ones, there are variant spaghetti diagrams for all the
  modelled systems.  The online supplement displays all the models
  presented for discussion during this work.
\label{fig:splinput}}
\end{figure*}

\begin{figure*}
\inclGrid{kappa_map_interpol}
\caption{Mass distribution $\kappa$ in the systems from
  Figures~\ref{fig:splinput}.
\label{fig:kappa}}
\end{figure*}

\begin{figure*}
\inclGrid{kappa_encl}
\caption{Enclosed mass within a given projected radius, expressed as
  the mean $\kappa$ with a given number of arcsec from the centre of
  the lensing galaxy.  The systems are the same as in
  Figures~\ref{fig:splinput}--\ref{fig:kappa}.  The orange band in
  each panel refers to the full ensemble of mass maps in the model,
  while the red curve is the ensemble average.  The dashed vertical
  line indicates the notional Einstein radius, or where the mean
  enclosed $\kappa$ is unity.  The short vertical arrows indicate the
  positions of the images (maxima, saddle points and minima).
  \label{fig:encl}}
\end{figure*}


\begin{figure*}
\inclGrid{arrival_spaghetti}
\caption{Arrival-time surfaces of the systems from
  Figure~\ref{fig:splinput}--\ref{fig:encl}.
  \label{fig:arriv}}
\end{figure*}

\begin{figure*}
\inclGrid{nsynth}
\caption{Synthetic images of the systems from
  Figures~\ref{fig:splinput}--\ref{fig:arriv}.
  \label{fig:synth}}
\end{figure*}


\begin{figure*}
\includegraphics[width=\linewidth]{img/mlens_vs_mstel_one/mstel_vs_mtot_one}
\caption{Total mass in the model against the estimated stellar mass,
  alongside the values for the whole sample.  The lower-right shaded
  region is unphysical according to the stellar-population models,
  because it gives $M<\Mstel$. The upper-left shaded region is
  unphysical according to abundance matching, because it gives
  $M>\Mhalo$.  That is to say, the unshaded region is
  $0<\haloindex<1$. \label{fig:stelmass}}
\end{figure*}


\begin{table*}
  \caption{Categorisation of SW models}
  \label{tab:models}
  
\begin{tabular}{c c c | c | c c c | c c c}
  \hline
  SWID & ASW id & model id
    & \rot{\shortstack[l]{spaghetti\\type}}
    
    & \multicolumn{1}{|l|}{\rot{\shortstack[l]{blending\\into arc}}}
    & \rot{\shortstack[l]{unseen\\counter\\image}}
    & \rot{\shortstack[l]{lensing\\pertubation}}
    
    & \rot{\shortstack[l]{img reconstr\\reasonable}}
    & \rot{\shortstack[l]{mass reconstr\\reasonable}}
    & \rot{M/L ratio}
  \\ \hline
 
% this is an example entry
%  SW99 & ASW000XXXX & 012345 & 0+0
%    & \NO & \NO & \OK
%    & \OK & \NO & \NO \\

  SW01 & ASW0004dv8 & 
    & X+X
    & \NO & \NO & \NO
    & \NO & \NO & \NO \\
    
  SW02 & ASW000619d & 011489
    & X+X
    & \NO & \NO & \NO
    & \NO & \NO & \NO \\
    
  SW03 & ASW0006mea & 
    & X+X
    & \NO & \NO & \NO
    & \NO & \NO & \NO \\
    
  SW04 & ASW0009cjs & NJ5CC5YJAQ
    & X+X
    & \NO & \NO & \NO
    & \NO & \NO & \NO \\
    
  SW05 & ASW0007k4r & AJIBCHQ6EM
    & 1+2+1
    & \NO & \NO & \NO
    & \OK & \OK & $\sim100$ \\
    
  SW06 & ASW0008swn & BCY2NOUSLK
    & X+X
    & \NO & \NO & \NO
    & \NO & \NO & \NO \\
    
  SW07 & ASW0007e08 & 
    & X+X
    & \NO & \NO & \NO
    & \NO & \NO & \NO \\
    
  SW08 & ASW00099ed & HISGRAIZL2
    & X+X
    & \NO & \NO & \NO
    & \NO & \NO & \NO \\
    
  SW09 & ASW0002asp & 5EKMWWVJHL
    & X+X
    & \NO & \NO & \NO
    & \NO & \NO & \NO \\
    
  SW10 & ASW0002bmc & VQYCYNONVW
    & X+X
    & \NO & \NO & \NO
    & \NO & \NO & \NO \\
    
  SW11 & ASW0002qtn & 3TUJKHGED4
    & X+X
    & \NO & \NO & \NO
    & \NO & \NO & \NO \\
    
  SW12 & ASW0003wsu & 012712
    & X+X
    & \NO & \NO & \NO
    & \NO & \NO & \NO \\
    
  SW13 & ASW00047ae & TGTIIF7HCV
    & X+X
    & \NO & \NO & \NO
    & \NO & \NO & \NO \\
    
  SW14 & ASW0004xjk & 
    & X+X
    & \NO & \NO & \NO
    & \NO & \NO & \NO \\
    
  SW15 & ASW0004nan & QUOGDU2NN6
    & X+X
    & \NO & \NO & \NO
    & \NO & \NO & \NO \\
    
  SW16 & ASW0009bp2 & 013421
    & X+X
    & \NO & \NO & \NO
    & \NO & \NO & \NO \\
    
  SW17 & ASW0005rnb & AAKHMTYTMS
    & X+X
    & \NO & \NO & \NO
    & \NO & \NO & \NO \\
    
  SW18 & ASW0007hu2 & D4UQI6M3ZU
    & X+X
    & \NO & \NO & \NO
    & \NO & \NO & \NO \\
    
  SW19 & ASW0001ld7 & OS3CYAKLRT
    & X+X
    & \NO & \NO & \NO
    & \NO & \NO & \NO \\
    
  SW20 & ASW0002dx7 & 3NYJG67KRT
    & X+X
    & \NO & \NO & \NO
    & \NO & \NO & \NO \\
    
  SW21 & ASW0004m3x & QZROE23AUH
    & X+X
    & \NO & \NO & \NO
    & \NO & \NO & \NO \\
    
  SW22 & ASW0009ab8 & TGM4U2TZBS
    & X+X
    & \NO & \NO & \NO
    & \NO & \NO & \NO \\
    
  SW23 & ASW0003r61 & 002481
    & X+X
    & \NO & \NO & \NO
    & \NO & \NO & \NO \\
    
  SW24 & ASW00050sk & 013406
    & X+X
    & \NO & \NO & \NO
    & \NO & \NO & \NO \\
    
  SW25 & ASW00007mq & 
    & X+X
    & \NO & \NO & \NO
    & \NO & \NO & \NO \\
    
  SW26 & ASW0005ma2 & 5ZZKUM3SWL
    & X+X
    & \NO & \NO & \NO
    & \NO & \NO & \NO \\
    
  SW27 & ASW0006jh5 & 5URN3BQFSV
    & X+X
    & \NO & \NO & \NO
    & \NO & \NO & \NO \\
    
  SW28 & ASW0007xrs & JHC3J2HYV7
    & X+X
    & \NO & \NO & \NO
    & \NO & \NO & \NO \\
    
  SW29 & ASW0008qsm & TOFS7JNGEK
    & X+X
    & \NO & \NO & \NO
    & \NO & \NO & \NO \\
    
  SW30 & ASW0002p8y & 
    & X+X
    & \NO & \NO & \NO
    & \NO & \NO & \NO \\
    
  SW31 & ASW00021r0 & SYTNGELH3Q
    & X+X
    & \NO & \NO & \NO
    & \NO & \NO & \NO \\
    
  SW32 & ASW0004iye & 
    & X+X
    & \NO & \NO & \NO
    & \NO & \NO & \NO \\
    
  SW33 & ASW0003s0m & ECXCIRBDUJ
    & X+X
    & \NO & \NO & \NO
    & \NO & \NO & \NO \\
    
  SW34 & ASW00051ld & 000291
    & X+X
    & \NO & \NO & \NO
    & \NO & \NO & \NO \\
    
  SW35 & ASW0004wgd & VWJ2LNN3VZ
    & X+X
    & \NO & \NO & \NO
    & \NO & \NO & \NO \\
    
  SW36 & ASW000096t & 7IPP7LWVOF
    & X+X
    & \NO & \NO & \NO
    & \NO & \NO & \NO \\
    
  SW37 & ASW00086xq & 
    & X+X
    & \NO & \NO & \NO
    & \NO & \NO & \NO \\
    
  SW38 & ASW0009cp0 & Z6IFI4SLLM
    & X+X
    & \NO & \NO & \NO
    & \NO & \NO & \NO \\
    
  SW39 & ASW0005qiz & 
    & X+X
    & \NO & \NO & \NO
    & \NO & \NO & \NO \\
    
  SW40 & ASW0008wmr & 
    & X+X
    & \NO & \NO & \NO
    & \NO & \NO & \NO \\
    
  SW41 & ASW0008xbu & BFXRMIQEAT
    & X+X
    & \NO & \NO & \NO
    & \NO & \NO & \NO \\
    
  SW42 & ASW00096rm & 4Q3YCEWGLN
    & X+X
    & \NO & \NO & \NO
    & \NO & \NO & \NO \\
    
  SW43 & ASW0001c3j & 5R6UYQZUTI
    & X+X
    & \NO & \NO & \NO
    & \NO & \NO & \NO \\
    
  SW44 & ASW0002k40 & 000899
    & X+X
    & \NO & \NO & \NO
    & \NO & \NO & \NO \\
    
  SW45 & ASW00024id & TSKKYHD3CB
    & X+X
    & \NO & \NO & \NO
    & \NO & \NO & \NO \\
    
  SW46 & ASW00024q6 & 012523
    & X+X
    & \NO & \NO & \NO
    & \NO & \NO & \NO \\
    
  SW47 & ASW0003r6c & 4HC3CREEAD
    & X+X
    & \NO & \NO & \NO
    & \NO & \NO & \NO \\
    
  SW48 & ASW0000g95 & 
    & X+X
    & \NO & \NO & \NO
    & \NO & \NO & \NO \\
    
  SW49 & ASW00007ls & 
    & X+X
    & \NO & \NO & \NO
    & \NO & \NO & \NO \\
    
  SW50 & ASW00008a0 & 
    & X+X
    & \NO & \NO & \NO
    & \NO & \NO & \NO \\
    
  SW51 & ASW0006e0o & 
    & X+X
    & \NO & \NO & \NO
    & \NO & \NO & \NO \\
    
  SW52 & ASW0006a07 & 
    & X+X
    & \NO & \NO & \NO
    & \NO & \NO & \NO \\
    
  SW53 & ASW00070vl & BPAV4GVOPP
    & X+X
    & \NO & \NO & \NO
    & \NO & \NO & \NO \\
    
  SW54 & ASW0007sez & SI4ELBAKL2
    & X+X
    & \NO & \NO & \NO
    & \NO & \NO & \NO \\
    
  SW55 & ASW0007t5y & 
    & X+X
    & \NO & \NO & \NO
    & \NO & \NO & \NO \\
    
  SW56 & ASW0007pga & VHV6RQYYKZ
    & X+X
    & \NO & \NO & \NO
    & \NO & \NO & \NO \\
    
  SW57 & ASW0008pag & 5SXGXQYY6V
    & X+X
    & \NO & \NO & \NO
    & \NO & \NO & \NO \\
    
  SW58 & ASW0007iwp & 4XBJWT3COV
    & X+X
    & \NO & \NO & \NO
    & \NO & \NO & \NO \\
    
  SW59 & ASW00085cp & 
    & X+X
    & \NO & \NO & \NO
    & \NO & \NO & \NO \\
    


  \hline

\end{tabular}

\end{table*}


\clearpage

\appendix

\section{Developments in SpaghettiLens}

\subsection{Improved synthetic images}

The mass maps produced by current implementation of SpaghettiLens are
based on images of point-like features.  No information about extended
images is used, except in so far as they help the user identify images
of point-like features.  The synthetic images offered to users are
rudimentary, corresponding to conical light profiles (that is,
circular light profiles with brightness decreasing linearly with radius).

We have now developed a prototype of better way to generate synthetic
images.  Figure~\figref{synthimg} illustrates.  First, areas
containing lensed images are selected (green frames in the figure).
The selected areas should be as free as possible from of light from
the lensing galaxy or from extraneous objects.  Pixels within the
selected areas are mapped to a grid on the source plane, using bending
angles given by the mass model.  The mapping from lens-plane pixels to
source-plane grid cells is many-to-one, because of image multiplicity
and magnification.  The brightness of each source-plane pixel is set
to the mean of all the lens-plane pixels mapping to it.  Finally, the
mapping is run back to the lens plane.  The result is a synthetic
image.  In effect, one is reconstructing a source-plane brightness map
by least-squares.

The procedure is not yet implemented in SpaghettiLens but can be
applied in post-processing.  The new synthetic images could be used to
improve the mass reconstruction, by weighting the ensemble of maps
according to how good the synthetic images are, but we have not
attempted to do so as yet.

\begin{figure}
  \includegraphics[width=\linewidth]{img/new_synth_img_detailed}
  \caption{Synthetic lensed image with source-profile fitting in SW05
    (J143454.4+522850). Top-left: original image, with areas
    containing lensed images enclosed within green frames.  Top-right:
    synthetic image (coloured arcs) with lensing galaxy and unrelated
    objects in greyscale.  Bottom from left to right: reconstructed
    source in colour, intensity (greyscale), count of lens plane
    pixels per source plane pixel, residual of original image to
    synthetic image.}
  \label{fig:synthimg}
\end{figure}

\subsection{Sub-sampling of central region}\label{subsec:hires}

The models of simulated lenses in \cite{2015MNRAS.447.2170K} showed a
tendency to be too shallow.  Allowing smaller mass tiles in the
central region, thus allowing the mass profile to rise more steeply
near the centre, was suggested as a possible cure.

Figure~\figref{subsampling} shows an experiment with smaller mass
tiles in the inner region.  Replacing the very central mass tile with
9 smaller tiles allows for steeper central profiles.  Doing the same
for the 25 innermost mass tiles solves the problem completely, but
increases the number of mass tiles by 40\% and significantly increases
the computational time.
% 689 / 489 = 1.40899; in practice, runtime is more like x4 - x5!
% runtimes from logfiles: 87/22, 86/21, 85/16
The main modelling work in this paper was, however, done before the
experiments with smaller mass tiles was complete.  Some of the models
presented in this paper apply the intermediate option (corresponding
to the middle panel in Figure~\figref{subsampling}) while others use
the old system.  The results in this paper, however, mainly concern
the enclosed mass in the outer regions, so shallowness in the central
region should be inconsequential.

\begin{figure}
  \includegraphics[width=.9\linewidth]{img/hires_comparison/ASW000102p_6941_11_hires_comparison}
  \includegraphics[width=.9\linewidth]{img/hires_comparison/ASW000102p_6941_13_hires_comparison}
  \includegraphics[width=.9\linewidth]{img/hires_comparison/ASW000102p_6941_33_hires_comparison}
  \caption{Model improvement resulting from using smaller mass tiles
    in the inner region of the mass model.  Shown here are the average
    enclosed $\kappa$ within a given projected radius, for three
    different reconstructions of a simulated lens (sim) from
    Space~Warps.  In each panel, the dashed blue curve is the correct
    answer.  The orange band represents the statistical ensemble from
    SpaghettiLens, the orange line being the ensemble mean.  Locations of
    images (maximum, saddle point, minimum) are marked with vertical
    arrows.  Crossing the horizontal $\kappa=1$ line is the effective
    Einstein radius \ER, and is so labelled. The upper panel is from
    K\"ung et al., (2015) (see Figure~3 of that paper).  The middle
    panel is the result when the innermost mass tile is replaced by 9
    smaller tiles.  The lower panel results from replacing each of the
    innermost 5 by 5 tiles each with 9 smaller tiles.}
  \label{fig:subsampling}
\end{figure}

\subsection{Parameterisation of pixel models} \label{subsec:parameter}

In order to fit the set of pixelated models to a single parameterised model, a program was written that took a parameterised function and subtracted from it the mean and the principal components of the data, which were calculated using classical Principal Component Analysis.
This created the residuals function.
The number of components used in the analysis was varied, to test how this affected the output, and it was found that using 5 principle components tended to give a reasonable approximation.
A masking function was added which selected only the data points that fell inside the image of the lens, and the principal components were clipped in order to keep the values inside the region of the ensemble of models.
Any value higher than the clip was set to be the clip value.
This was chosen to be 2.5 as, assuming that the data follows a Gaussian error distribution, almost all the values for the variance should lie between 2 and 3 standard deviations from the mean.
Minimising the residuals function produces the set of parameters that fit the parameterised function to the original pixelated ensemble most closely.
A least squares fit was used to perform this minimisation.
The parameterised model function was obtained from the gravitational potential of an isothermal ellipsoid mass distribution \citep{2001astro.ph..2341K}.
This model is frequently used to describe gravitational lenses as it tends to fit well with observations.
The isothermal ellipsoid model outputs three useful parameters: the radius of the Einstein ring, the ellipticity of the model and the angle of the ellipticity from the vertical, giving the orientation of the galaxy.
By applying this model to simulated lenses for which the values of these parameters were already known, it was possible to gain an estimate of the projected accuracy of the results, before applying the model to the candidate lensing galaxies.

Preliminary results on recovery of Einstein radii are shown in
Figure~\ref{fig:parameter}. \todo{Einstein radii from parametric
  model-fitting.}

\begin{figure}
  \includegraphics[width=\linewidth]{img/rE_comp/rE_comp.png}
  \caption{Comparison of Einstein radii. {\em To be extended.}}
  \label{fig:parameter}
\end{figure}






% Don't change these lines
\bsp	% typesetting comment
\label{lastpage}
\end{document}
