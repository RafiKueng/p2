% mnras_template.tex
%
% LaTeX template for creating an MNRAS paper
%
% v3.0 released 14 May 2015
% (version numbers match those of mnras.cls)
%
% Copyright (C) Royal Astronomical Society 2015
% Authors:
% Keith T. Smith (Royal Astronomical Society)

% Change log
%
% v3.0 May 2015
%    Renamed to match the new package name
%    Version number matches mnras.cls
%    A few minor tweaks to wording
% v1.0 September 2013
%    Beta testing only - never publicly released
%    First version: a simple (ish) template for creating an MNRAS paper

%%%%%%%%%%%%%%%%%%%%%%%%%%%%%%%%%%%%%%%%%%%%%%%%%%
% Basic setup. Most papers should leave these options alone.
\documentclass[a4paper,fleqn,usenatbib]{mnras}

% MNRAS is set in Times font. If you don't have this installed (most LaTeX
% installations will be fine) or prefer the old Computer Modern fonts, comment
% out the following line
%\usepackage{newtxtext,newtxmath} % not yet supported on arxiv and uni computer
%\usepackage{txfonts}

% Depending on your LaTeX fonts installation, you might get better results with one of these:
%\usepackage{mathptmx}
%\usepackage{txfonts}

% Use vector fonts, so it zooms properly in on-screen viewing software
% Don't change these lines unless you know what you are doing
\usepackage[T1]{fontenc}
\usepackage{ae,aecompl}


%%%%% AUTHORS - PLACE YOUR OWN PACKAGES HERE %%%%%

% Only include extra packages if you really need them. Common packages are:
\usepackage{graphicx}	% Including figure files
\usepackage{amsmath}	% Advanced maths commands
\usepackage{amssymb}	% Extra maths symbols

%%%%%%%%%%%%%%%%%%%%%%%%%%%%%%%%%%%%%%%%%%%%%%%%%%

%%%%% AUTHORS - PLACE YOUR OWN COMMANDS HERE %%%%%

% Please keep new commands to a minimum, and use \newcommand not \def to avoid
% overwriting existing commands. Example:
%\newcommand{\pcm}{\,cm$^{-2}$}	% per cm-squared


\newcommand{\inclfig}[2]{
  \includegraphics[width=0.49\linewidth]{spaghetti/#2_input.png}
  \includegraphics[width=0.49\linewidth]{spaghetti/#2_img1.png}
  \includegraphics[width=0.49\linewidth]{spaghetti/#2_img3_ipol.png}
  \includegraphics[width=0.49\linewidth]{spaghetti/#2_img2.png}
  \includegraphics[width=1\linewidth]{masses/#1_#2_mstel_vs_mtot.png}
}

\newcommand{\lenstitle}[1]{\noindent\textit{#1}\quad}



%%%%%%%%%%%%%%%%%%%%%%%%%%%%%%%%%%%%%%%%%%%%%%%%%%

%%%%%%%%%%%%%%%%%%% TITLE PAGE %%%%%%%%%%%%%%%%%%%

% Title of the paper, and the short title which is used in the headers.
% Keep the title short and informative.
\title[Short title, max. 45 characters]{Preliminary categorisation of SpaceWarps candidates}

% The list of authors, and the short list which is used in the headers.
% If you need two or more lines of authors, add an extra line using \newauthor
\author[R. Kueng et al.]{
Rafael Kueng,$^{1}$\thanks{E-mail: rafael.kueng@uzh.ch}
Presenjit Saha,$^{1}$
Lucy Oswald$^{3}$
%and Fourth Author$^{4}$
\\
% List of institutions
$^{1}$Physik Institut, University of Zurich, Winterthurerstrasse XXX, Zurich 8057, Switzerland\\
$^{2}$Department, Institution, Street Address, City Postal Code, Country\\
$^{3}$Another Department, Different Institution, Street Address, City Postal Code, Country
%$^{4}$Another Department, Different Institution, Street Address, City Postal Code, Country
}

% These dates will be filled out by the publisher
\date{Accepted XXX. Received YYY; in original form ZZZ}

% Enter the current year, for the copyright statements etc.
\pubyear{2016}



% Don't change these lines
\begin{document}
\label{firstpage}
\pagerange{\pageref{firstpage}--\pageref{lastpage}}
\maketitle

% Abstract of the paper
\begin{abstract}

%This is a simple template for authors to write new MNRAS papers.
%The abstract should briefly describe the aims, methods, and main results of the paper.
%It should be a single paragraph not more than 250 words (200 words for Letters).
%No references should appear in the abstract.
\end{abstract}

% Select between one and six entries from the list of approved keywords.
% Don't make up new ones.
\begin{keywords}
keyword1 -- keyword2 -- keyword3
\end{keywords}

%%%%%%%%%%%%%%%%%%%%%%%%%%%%%%%%%%%%%%%%%%%%%%%%%%
%%%%%%%%%%%%%%%%% BODY OF PAPER %%%%%%%%%%%%%%%%%%
%%%%%%%%%%%%%%%%%%%%%%%%%%%%%%%%%%%%%%%%%%%%%%%%%%

\section{Introduction}

Here we cite \cite{2015MNRAS.447.2170K}.

\section{Methods}

\subsection{Parameterisation of pixel models}
In order to fit the set of pixelated models to a single parameterised model, a program was written that took a parameterised function and subtracted from it the mean and the principle components of the data, which were calculated using classical Principle Component Analysis.
This created the residuals function.
The number of components defined as principle was varied to test how this affected the output, and it was found that using 5 principle components tended to give a reasonable approximation.
A masking function was added which selected only the data points that fell inside the image of the lens, and the principal components were clipped in order to keep the values inside the region of the ensemble of models.
Any value higher than the clip was set to be the clip value.
This was chosen to be 2.5 as, assuming that the data follows a Gaussian error distribution, almost all the values for the variance should lie between 2 and 3 standard deviations from the mean.
Minimising the residuals function produces the set of parameters that fit the parameterised function to the original pixelated ensemble most closely.
A least squares fit was used to perform this minimisation.


The parameterised model function was obtained from the gravitational potential of an isothermal ellipsoid mass distribution \cite{2001astro.ph..2340K}.
This model is frequently used to describe gravitational lenses as it tends to fit well with observations.
The isothermal ellipsoid model outputs three useful parameters: the radius of the Einstein ring, the ellipticity of the model and the angle of the ellipticity from the vertical, giving the orientation of the galaxy.
By applying this model to simulated lenses for which the values of these parameters were already known, it was possible to gain an estimate of the projected accuracy of the results, before applying the model to the candidate lensing galaxies.


%TODO append this to the bibliography file (bibli.bib)
% References
% Keeton C.
% R., 2002, “A Catalog of Mass Models for Gravitational Lensing”, arXiv:astro-ph/0102341v2



\section{Results}
We came up with four kind of candidates.

\subsection{HiRes image}
Almost certain candidates, that qualify for follow up hires observations



\subsection{Spectra}
Candidates to follow up with a spectral analysis. 


\subsection{Bad models}
Candidates that need more work to be modelled.
Or that have features that are hard / impossible to model with the current SpaghettiLens version.
Examples are group lenses



\subsection{Non Lenses}
Those are most probly non lenses, 



\subsection{plausible}

\lenstitle{SW01 (ASW0004dv8)}
	This is some text that says somwthing more than this line.

\lenstitle{SW08 (ASW00099ed)}
  Simple lens with two images;
  model has minor arthefacts.

\lenstitle{SW11 (ASW0002qtn)}
  Simple small arc;
  model nedes minor refinement;
  quite low mass ratio
  
\begin{itemize}
  \item SW12 (ASW0003wsu) \\
  faint arc, bright overall situation, no counter image;
  has good model;
  low Mstel and low Mlens
  
  \item SW21 (ASW0004m3x) \\
  clear arc, no counter image; inbetween two external sources;
  model needs work, predicts additional images. play with external mass / sheer?
  Mass ratio low.
  
  \item SW27 (ASW0006jh5) \\
  (really plausible?)
  unclear situation, but arc is visible;
  model shows extended arc not there in org image;
  mass ratio ok
  
  \item SW31 (ASW00021r0) \\
  (really plausible?)
  noisy situation, unclear images, potential external mass interfering;
  model ok'ish, hard to say;
  high mass ratio
  
  \item SW33 (ASW0003s0m)\\
  clear picture, two image situation;
  model simple, but ok. arc too much extended;
  masses in the middle, but rather high ratio
  
  \item SW36 (ASW000096t) \\
  noisy picture, many possible external masses, faint images;
  model predicts additional images, needs more work, arc too extended;
  masses look ok
  
  \item SW38 (ASW0009cp0) \\
  quite clean picture, two bright minima, counter image hidden;
  model fine
  really hi stellar mass
  
  \item SW43 (ASW0001c3j) \\
  clean situation, faint arc'ish images;
  model needs work, shows additional images, strange arc (LHD);
  very high Mstel
  
  \item SW45 (ASW00024id)\\
  ring like structure, very clean \\
  usual problems with rings, otherwise model quite ok actually;
  very high Mstel
  
  \item SW46 (ASW00024q6) \\
  yellow'ish arc, two image configuration;
  model ok;
  masses ok
  
  \item SW52 (ASW0006a07) \\
  (no imgs)
  
  \item SW54 (ASW0007sez) \\
  white lens, fuzzy ring like structure;
  model really good, given ring like;
  masses ok
  
  \item SW58 (ASW0007iwp) \\
  nice arc, no counter image;
  model prediction really good;
  masses ok
  
\end{itemize}

good example of plausible

% \begin{figure}
% 	\inclfig{SW45}{ASW000024id_TSKKYHD3CB}
% \end{figure}

\begin{figure}
	\inclfig{SW58}{ASW0007iwp_4XBJWT3COV}
\end{figure}


\subsubsection{grouplens}

\begin{figure}
	\inclfig{SW36}{ASW000096t_7IPP7LWVOF}
\end{figure}

(like 4dv8 collective modelling)




\subsection{unclear}
\begin{itemize}
  \item SW04 (ASW0009cjs) \\
  noisy field, possible group lens, but images not typical? (no arclike structure);
  model fail (input errors);
  Mlens too high (10e13)
  
  \item SW06 (ASW0008swn)\\
  green images, elongated lens, images on strange positions?;
  model ok'ish, but mass distr. doesn't reflect visuals;
  masses ok
  
  \item SW07 (ASW0007e08) \\
  (no img)
  
  \item SW13 (ASW00047ae) \\
  croweded situation unclear setup;
  model failed, predicts additional images;
  masses ok
  
  \item SW15 (ASW0004nan) \\
  no counter image (should be visible)?
  model ok'ish;
  Mlens on the lower end
  
  \item SW17 (ASW0005rnb) \\
  two image setup, but arcish structure of images missing;
  model failed, predicts additional images, rednering failed;
  
  \item SW19 (ASW0001ld7) \\
  asymmetry / really extended arc and no counter image rather no lensing configuration;
  rendering ok'ish, model predicts additionla images;
  Mstel and Mlens way too low (e10 / e11)
  
  \item SW24 (ASW00050sk) \\
  unclear / noisy situation, shape of arc strange;
  model ok;
  mass ration close to unity
  
  \item SW32 (ASW0004iye) \\
  (no img)
  
  \item SW35 (ASW0004wgd) \\
  hard to say, doesn't look convincing, but neither totally off
  modell ok;
  masses as well;
  
  \item SW44 (ASW0002k40) \\
  (no img);
  very old model;
  hi mass ratio
  
  \item SW47 (ASW0003r6c) \\
  two images, no arcish structre.. too far away from lens?;
  model predicts additional images;
  quite high mass ratio
  
  \item SW51 (ASW0006e0o) \\
  (no img)
  
  \item SW56 (ASW0007pga) \\
  very ellyptical lens, images not really arcish;
  model failed, predicts additional images;
  masses ok
  
  \item SW57 (ASW0008pag) \\
  point mass / group lens interfering? symertry of arc off;
  model failed completly;
  Mlens too high
  
\end{itemize}


\begin{figure}
	\inclfig{SW19}{ASW0001ld7_OS3CYAKLRT}
\end{figure}


\begin{figure}
	\inclfig{SW57}{ASW0008pag_5SXGXQYY6V}
\end{figure}



\subsection{convincing}
\begin{itemize}
  \item SW02 (ASW000619d) \\
  
    
  \item SW05 (ASW0007k4r) \\
  
  \item SW09 (ASW0002asp) \\
  
  \item SW28 (ASW0007xrs) \\
  
  \item SW29 (ASW0008qsm) \\
  
\end{itemize}

\begin{figure}
	\inclfig{SW02}{ASW000619d_011489}
\end{figure}

\begin{figure}
	\inclfig{SW05}{ASW0007k4r_AJIBCHQ6EM}
\end{figure}

\begin{figure}
	\inclfig{SW09}{ASW0002asp_5EKMWWVJHL}
\end{figure}

\begin{figure}
	\inclfig{SW28}{ASW0007xrs_JHC3J2HYV7}
\end{figure}

\begin{figure}
	\inclfig{SW29}{ASW0008qsm_TOFS7JNGEK}
\end{figure}




\subsection{doubtful}
\begin{itemize}
  \item SW10 (ASW0002bmc) \\
  very elliptical setup, unlikley to be a lens;
  model failed totally
  
  \item SW16 (ASW0009bp2) \\
  very faint images
  
  \item SW18 (ASW0007hu2) \\
  model failed completly
  
  \item SW20 (ASW0002dx7) \\
  faint white images, rather substuctre of ``lens'' ??
  
  \item SW22 (ASW0009ab8) \\
  very faint images, group configuration;
  model predicts additional images;
  mass ration close to unity
  
  \item SW23 (ASW0003r61) \\
  (no img)
  
  \item SW26 (ASW0005ma2) \\
  similar to SW10
  
  \item SW34 (ASW00051ld) \\
  (no img)
  
  \item SW41 (ASW0008xbu) \\
  similar to SW10. very ellyptical lens
  
  \item SW42 (ASW00096rm) \\
  ring like structure, but very strange distribution;
  model not too bad, brightnesses off;
  stellar mass way too lower
  
  \item SW53 (ASW00070vl) \\
  images don't look liked lensed;
  model predicts additional images
  
\end{itemize}



outlier because very odd mass ratio
\begin{figure}
	\inclfig{SW42}{ASW00096rm_4Q3YCEWGLN}
\end{figure}







\subsection{missing}
\begin{itemize}
  \item SW03 (ASW0006mea)
  \item SW14 (ASW0004xjk)
  \item SW25 (ASW00007mq)
  \item SW30 (ASW0002p8y)
  \item SW37 (ASW00086xq)
  \item SW39 (ASW0005qiz)
  \item SW40 (ASW0008wmr)
  \item SW48 (ASW0000g95)
  \item SW49 (ASW00007ls)
  \item SW50 (ASW00008a0)
  \item SW55 (ASW0007t5y)
  \item SW59 (ASW00085cp)
\end{itemize}





%%%%%%%%%%%%%%%%%%%%%%%%%%%%%%%%%%%%%%%%%%%%%%%%%%
%%%%%%%%%%%%%%%%%%%% REFERENCES %%%%%%%%%%%%%%%%%%
%%%%%%%%%%%%%%%%%%%%%%%%%%%%%%%%%%%%%%%%%%%%%%%%%%

% The best way to enter references is to use BibTeX:
\bibliographystyle{mnras}
\bibliography{bib/bibli} % if your bibtex file is called example.bib


%%%%%%%%%%%%%%%%%%%%%%%%%%%%%%%%%%%%%%%%%%%%%%%%%%
%%%%%%%%%%%%%%%%% APPENDICES %%%%%%%%%%%%%%%%%%%%%
%%%%%%%%%%%%%%%%%%%%%%%%%%%%%%%%%%%%%%%%%%%%%%%%%%

\appendix

\section{Some extra material}

If you want to present additional material which would interrupt the flow of the main paper,
it can be placed in an Appendix which appears after the list of references.

%%%%%%%%%%%%%%%%%%%%%%%%%%%%%%%%%%%%%%%%%%%%%%%%%%



% Don't change these lines
\bsp	% typesetting comment
\label{lastpage}
\end{document}



%%%%%%%%%%%%%%%%%%%%%%%%%%%%%%%%%%%%%%%%%%%%%%%%%%
%%%%%%%%%%%%%%%%%%%% END %%%%%%%%%%%%%%%%%%%%%%%%%
%%%%%%%%%%%%%%%%%%%%%%%%%%%%%%%%%%%%%%%%%%%%%%%%%%








% 
% \section{Introduction}
% 
% This is a simple template for authors to write new MNRAS papers.
% See \texttt{mnras\_sample.tex} for a more complex example, and \texttt{mnras\_guide.tex}
% for a full user guide.
% 
% All papers should start with an Introduction section, which sets the work
% in context, cites relevant earlier studies in the field by \citet{Others2013},
% and describes the problem the authors aim to solve \citep[e.g.][]{Author2012}.
% 
% \section{Methods, Observations, Simulations etc.}
% 
% Normally the next section describes the techniques the authors used.
% It is frequently split into subsections, such as Section~\ref{sec:maths} below.
% 
% \subsection{Maths}
% \label{sec:maths} % used for referring to this section from elsewhere
% 
% Simple mathematics can be inserted into the flow of the text e.g. $2\times3=6$
% or $v=220$\,km\,s$^{-1}$, but more complicated expressions should be entered
% as a numbered equation:
% 
% \begin{equation}
%     x=\frac{-b\pm\sqrt{b^2-4ac}}{2a}.
% 	\label{eq:quadratic}
% \end{equation}
% 
% Refer back to them as e.g. equation~(\ref{eq:quadratic}).
% 
% \subsection{Figures and tables}
% 
% Figures and tables should be placed at logical positions in the text. Don't
% worry about the exact layout, which will be handled by the publishers.
% 
% Figures are referred to as e.g. Fig.~\ref{fig:example_figure}, and tables as
% e.g. Table~\ref{tab:example_table}.
% 
% % Example figure
% \begin{figure}
% 	% To include a figure from a file named example.*
% 	% Allowable file formats are eps or ps if compiling using latex
% 	% or pdf, png, jpg if compiling using pdflatex
% 	\includegraphics[width=\columnwidth]{example}
%     \caption{This is an example figure. Captions appear below each figure.
% 	Give enough detail for the reader to understand what they're looking at,
% 	but leave detailed discussion to the main body of the text.}
%     \label{fig:example_figure}
% \end{figure}
% 
% % Example table
% \begin{table}
% 	\centering
% 	\caption{This is an example table. Captions appear above each table.
% 	Remember to define the quantities, symbols and units used.}
% 	\label{tab:example_table}
% 	\begin{tabular}{lccr} % four columns, alignment for each
% 		\hline
% 		A & B & C & D\\
% 		\hline
% 		1 & 2 & 3 & 4\\
% 		2 & 4 & 6 & 8\\
% 		3 & 5 & 7 & 9\\
% 		\hline
% 	\end{tabular}
% \end{table}
% 
% 
% \section{Conclusions}
% 
% The last numbered section should briefly summarise what has been done, and describe
% the final conclusions which the authors draw from their work.
% 
% \section*{Acknowledgements}
% 
% The Acknowledgements section is not numbered. Here you can thank helpful
% colleagues, acknowledge funding agencies, telescopes and facilities used etc.
% Try to keep it short.
% 
% %%%%%%%%%%%%%%%%%%%%%%%%%%%%%%%%%%%%%%%%%%%%%%%%%%
% 
% %%%%%%%%%%%%%%%%%%%% REFERENCES %%%%%%%%%%%%%%%%%%
% 
% % The best way to enter references is to use BibTeX:
% 
% %\bibliographystyle{mnras}
% %\bibliography{example} % if your bibtex file is called example.bib
% 
% 
% % Alternatively you could enter them by hand, like this:
% % This method is tedious and prone to error if you have lots of references
% \begin{thebibliography}{99}
% \bibitem[\protect\citeauthoryear{Author}{2012}]{Author2012}
% Author A.~N., 2013, Journal of Improbable Astronomy, 1, 1
% \bibitem[\protect\citeauthoryear{Others}{2013}]{Others2013}
% Others S., 2012, Journal of Interesting Stuff, 17, 198
% \end{thebibliography}
% 
% %%%%%%%%%%%%%%%%%%%%%%%%%%%%%%%%%%%%%%%%%%%%%%%%%%
% 
% %%%%%%%%%%%%%%%%% APPENDICES %%%%%%%%%%%%%%%%%%%%%
% 
% \appendix
% 
% \section{Some extra material}
% 
% If you want to present additional material which would interrupt the flow of the main paper,
% it can be placed in an Appendix which appears after the list of references.
% 
% %%%%%%%%%%%%%%%%%%%%%%%%%%%%%%%%%%%%%%%%%%%%%%%%%%
% 
% 
% % Don't change these lines
% \bsp	% typesetting comment
% \label{lastpage}
% \end{document}
% 
% % End of mnras_template.tex
