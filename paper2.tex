\documentclass[fleqn,usenatbib]{mnras}

%%% DEBUG SETTINGS - remove at the end %%%%%%%%%%%%%%%
\usepackage{silence} % silence error warinings
\WarningFilter{caption}{Unsupported document class} % caption doesn't know mnras..
\hbadness = 5000
\vbadness = 10000
% \usepackage{}

% MNRAS is set in Times font. If you don't have this installed (most LaTeX
% installations will be fine) or prefer the old Computer Modern fonts, comment
% out the following line
%\usepackage{newtxtext,newtxmath} % not yet supported on arxiv and uni computer
%\usepackage{txfonts}

% Depending on your LaTeX fonts installation, you might get better results with one of these:
%\usepackage{mathptmx}
%\usepackage{txfonts}

% Use vector fonts, so it zooms properly in on-screen viewing software
% Don't change these lines unless you know what you are doing
\usepackage[T1]{fontenc} % make sure font is supported
% \usepackage{ae,aecompl} %obsolete when using modern fonts
\usepackage[final]{microtype} % make sure font is supported
% and a suitable font
\usepackage{lmodern} % use a modern font with T1 support
%\usepackage{cm-super} % use a modern font with T1 support


%%%%% AUTHORS - PLACE YOUR OWN PACKAGES HERE %%%%%

% Only include extra packages if you really need them. Common packages are:
%\usepackage[dvipdfmx]{graphicx}	% Including figure files
\usepackage{graphicx} % Including figure files
%\usepackage[skip=0pt]{subcaption}
\usepackage{amsmath}	% Advanced maths commands
\usepackage{amssymb}	% Extra maths symbols
\usepackage{pifont}	% Extra maths symbols
%\usepackage{bmpsize}  % PS still needs this for correct bounding boxes?


% TODO stuff, comment out for final submit
\usepackage{todonotes} % as long as we are editing...
%\usepackage{showframe} % for debug
%\overfullrule=5pt      % show overfull boxes

% I disabled hyperlink in the cls und included it here because of bugs
\usepackage{hyperref}   % Hyperlinks
\hypersetup{colorlinks=true,linkcolor=blue,citecolor=blue,filecolor=blue,urlcolor=blue}

\edef\smallwidth{0.4\linewidth}

\newcommand{\inclfig}[2]{
  \centering
	\includegraphics[width=\smallwidth]{spaghetti/#2_input}%
	\includegraphics[width=\smallwidth]{spaghetti/#2_img3_ipol}
%	\includegraphics[width=\smallwidth]{spaghetti/#2_img1}%
        \includegraphics[width=\smallwidth]{img/arrival_spaghetti/#1_#2_arrival_spaghetti}%
%        \includegraphics[width=\smallwidth]{spaghetti/#2_img2}
        \includegraphics[width=\smallwidth]{img/kappa_map/#1_#2_kappa_map}
	\includegraphics[width=\smallwidth]{img/kappa_encl/#1_#2_kappa_encl}%
	\includegraphics[width=\smallwidth]{img/mlens_vs_mstel/#1_#2_mstel_vs_mtot}
}

\newcommand{\inclfign}[2]{
  \centering
	\includegraphics[width=\smallwidth]{spaghetti/#2_input}%
	\includegraphics[width=\smallwidth]{img/nsynth/#2_nsynth}
%        \includegraphics[width=\smallwidth]{spaghetti/#2_img1}%
        \includegraphics[width=\smallwidth]{img/arrival_spaghetti/#1_#2_arrival_spaghetti}%
%	\includegraphics[width=\smallwidth]{spaghetti/#2_img2}
        \includegraphics[width=\smallwidth]{img/kappa_map/#1_#2_kappa_map}
	\includegraphics[width=\smallwidth]{img/kappa_encl/#1_#2_kappa_encl}%
	\includegraphics[width=\smallwidth]{img/mlens_vs_mstel/#1_#2_mstel_vs_mtot}
}

\newcommand*{\rot}{\rotatebox{90}}
\newcommand*{\OK}{\ding{51}}
\newcommand*{\NO}{\ding{55}}

\newcommand{\lenstitle}[1]{\noindent\textbf{#1} --}
\newcommand{\params}[3]{(\(\Theta_\text{E}:#1\), $\varepsilon:#2$, $\alpha_\varepsilon:#3$ )}

\newcommand{\asw}[1]{ASW000#1}
\newcommand{\sw}[1]{SW~#1}
\newcommand{\model}[1]{SL model~#1}

\newcommand{\figref}[1]{\ref{fig:#1}}

\newcommand{\Mstel}{M_{\rm stel}}
\newcommand{\Mhalo}{M_{\rm h}}
\newcommand{\haloindex}{\mathcal{H}}
\newcommand{\ER}{$\Theta_{\text{E}}$} % einstein radius

\title[Lens models for Space Warps CFHTLS]{Models of lens candidates from
  Space Warps CFHTLS}

\author[K\"ung et al]{Rafael K\"ung,$^{1}$
Prasenjit Saha,$^{1}$
Ignacio Ferreras,$^{2}$
Elisabeth Baeten,$^{3}$
\newauthor
Jonathan Coles,$^{4}$
Claude Cornen,$^{3}$
Christine Macmillan,$^{3}$
Phil Marshall,$^{5}$ 
\newauthor
Anupreeta More,$^{6}$
Lucy Oswald$^{7}$
Aprajita Verma$^{8}$
and Julianne K. Wilcox$^{4}$
%
\\
%
$^{1}$Physik-Institut, University of Zurich, Winterthurerstrasse 190, 8057 Zurich, Switzerland\\
$^{2}$Mullard Space Science Laboratory, University College London, Holmbury St Mary, Dorking, Surrey RH5 6NT, UK\\
$^{3}$Zooniverse, c/o Astrophysics Department, University of Oxford, Oxford OX1 3RH, UK \\
$^{4}$Exascale Research Computing Lab, Campus Teratec, 2 Rue de la Piquetterie, 91680 Bruyeres-le-Chatel, France\\
$^{5}$Kavli Institute for Particle Astrophysics and Cosmology, Stanford University, 452 Lomita Mall, Stanford, CA 94035, USA\\
$^{6}$Kavli Institute for the Physics and Mathematics of the Universe, University of Tokyo, 5-1-5 Kashiwanoha, Kashiwa-shi 277-8583, Japan\\
$^{7}$Murray Edwards College, University of Cambridge, Cambridge CB3 0DF, UK\\
$^{8}$Sub-department of Astrophysics, University of Oxford, Denys Wilkinson Building, Keble Road, Oxford, OX1 3RH, UK\\
}



% These dates will be filled out by the publisher
% \date{Accepted XXX. Received YYY; in original form ZZZ}

% Enter the current year, for the copyright statements etc.
\pubyear{2016}

% Don't change these lines
\begin{document}
\label{firstpage}
\pagerange{\pageref{firstpage}--\pageref{lastpage}}
\maketitle

\begin{abstract}
We report modelling follow-up of recently-discovered
gravitational-lens candidates in the CFHT Legacy Survey.  Lens
modelling was done by a small group of specially-interested volunteers
from the Space~Warps citizen-science community who originally found
the candidate lenses.  Models are categorised according to seven
qualitative and quantitive points.  Also included are some
improvements to the modelling software used (SpaghettiLens),
and discussion of strategies for scaling to future surveys
with more and frequent discoveries.

The candidates successfully modelled are all galaxies, with inferred
lensing masses ranging from $\sim10^{11}M_\odot$ to $>10^{13}M_\odot$.
Stellar masses have also been estimated, using photometry from the
CFHTLS pipeline and stellar-population models.  A trend well-known
in nearby galaxies, that star-formation efficiency is maximal for
total masses of $\approx10^{12}M_\odot$ and reduces for both lower and
higher masses, is discernable also in these much more distant
galaxies.
\end{abstract}

\begin{keywords}
gravitational lensing: strong -- keyword2
\end{keywords}

\def\pwidth{.32\linewidth}

\def\includeten#1#2{
\includegraphics[width=\pwidth]{#1ASW0007k4r_N7LTELSYTM#2}%
\includegraphics[width=\pwidth]{#1ASW00096rm_4Q3YCEWGLN#2}%
\includegraphics[width=\pwidth]{#1ASW0007xrs_JHC3J2HYV7#2}\\
\includegraphics[width=\pwidth]{#1ASW0007iwp_4XBJWT3COV#2}%
\includegraphics[width=\pwidth]{#1ASW000619d_011489#2}%
\includegraphics[width=\pwidth]{#1ASW0001ld7_OS3CYAKLRT#2}\\
\includegraphics[width=\pwidth]{#1ASW0002asp_5EKMWWVJHL#2}%
%\includegraphics[width=\pwidth]{#1ASW000096t_7IPP7LWVOF#2}%
\includegraphics[width=\pwidth]{#1ASW0008qsm_TOFS7JNGEK#2}%
\includegraphics[width=\pwidth]{#1ASW0008pag_5SXGXQYY6V#2}\\
}

\def\includezehn#1#2{
\includegraphics[width=\pwidth]{#1SW05_ASW0007k4r_N7LTELSYTM#2}%
\includegraphics[width=\pwidth]{#1SW42_ASW00096rm_4Q3YCEWGLN#2}%
\includegraphics[width=\pwidth]{#1SW28_ASW0007xrs_JHC3J2HYV7#2}\\
\includegraphics[width=\pwidth]{#1SW58_ASW0007iwp_4XBJWT3COV#2}%
\includegraphics[width=\pwidth]{#1SW02_ASW000619d_011489#2}%
\includegraphics[width=\pwidth]{#1SW19_ASW0001ld7_OS3CYAKLRT#2}\\
\includegraphics[width=\pwidth]{#1SW09_ASW0002asp_5EKMWWVJHL#2}%
%\includegraphics[width=\pwidth]{#1SW36_ASW000096t_7IPP7LWVOF#2}\\
\includegraphics[width=\pwidth]{#1SW29_ASW0008qsm_TOFS7JNGEK#2}%
\includegraphics[width=\pwidth]{#1SW57_ASW0008pag_5SXGXQYY6V#2}\\
}

\input parts/intro

\input parts/spl

\input parts/morph

\input parts/massmodel

\input parts/stelmass


\begin{table*}
  \caption{Categorisation of SW models}
  \label{tab:models}
  
\begin{tabular}{c c c | c c | c c c | c c c}
  \hline
  SWID & ASW id & model id
  
    & \rot{$z_\text{lens}$}

    & \rot{\shortstack[l]{image\\morphology}}
    
    & \multicolumn{1}{|l|}{\rot{\shortstack[l]{unblended\\images}}}
    & \rot{\shortstack[l]{all images\\discernible}}
    & \rot{\shortstack[l]{isolated\\ lens}}
    
    & \rot{\shortstack[l]{synthetic image\\ reasonable}}
    & \rot{\shortstack[l]{mass map\\ reasonable}}
    & \rot{\shortstack[l]{total vs stellar\\ mass ratio}}
  \\ \hline
  SW01 & ASW0004dv8 & J022409.5-105807 & 
    & 
    &  &  & 
    &  &  &  \\
    
  SW02 & ASW000619d & J140522.2+574333 & 0.7
    & LQ
    & \NO & \OK & \NO
    & \OK & \OK & 10 \\
    
  SW03 & ASW0006mea & J142603.2+511421 & 
    & 
    &  &  & 
    &  &  &  \\
    
  SW04 & ASW0009cjs & J142934.2+562541 & 0.5
    & CQ
    & \OK & \NO & \NO
    & \NO & \OK & 74 \\
    
  SW05 & ASW0007k4r & J143454.4+522850 & 0.6
    & IQ
    & \OK & \OK & \OK
    & \OK & \OK & 1.0e+02 \\
    
  SW06 & ASW0008swn & J143627.9+563832 & 0.5
    & LQ
    & \NO & \OK & \OK
    & \OK & \NO & 7 \\
    
  SW07 & ASW0007e08 & J220256.8+023432 & 
    & 
    &  &  & 
    &  &  &  \\
    
  SW08 & ASW00099ed & J020648.0-065639 & 0.8
    & D
    & \OK & \OK & \NO
    & \OK & \OK & 7 \\
    
  SW09 & ASW0002asp & J020832.1-043315 & 1.0
    & SQ
    & \NO & \OK & \OK
    & \OK & \OK & 9 \\
    
  SW10 & ASW0002bmc & J020848.2-042427 & 0.8
    & D
    & \OK & \NO & \OK
    & \NO & \NO & 3 \\
    
  SW11 & ASW0002qtn & J020849.8-050429 & 0.8
    & LQ
    & \NO & \OK & \NO
    & \OK & \OK & 3 \\
    
  SW12 & ASW0003wsu & J022406.1-062846 & 0.4
    & D
    & \OK & \OK & \NO
    & \OK & \OK & 4 \\
    
  SW13 & ASW00047ae & J022805.6-051733 & 0.4
    & LQ
    & \NO & \NO & \NO
    & \NO & \NO & 7 \\
    
  SW14 & ASW0004xjk & J023123.2-082535 & 
    & 
    &  &  & 
    &  &  &  \\
    
  SW15 & ASW0004nan & J084841.0-045237 & 0.3
    & LQ
    & \NO & \OK & \NO
    & \OK & \OK & 8 \\
    
  SW16 & ASW0009bp2 & J140030.2+574437 & 0.4
    & D
    & \NO & \NO & \OK
    & \NO & \OK & 5 \\
    
  SW17 & ASW0005rnb & J140622.9+520942 & 0.7
    & D
    & \OK & \NO & \NO
    & \NO & \OK & 6 \\
    
  SW18 & ASW0007hu2 & J143658.1+533807 & 0.7
    & D
    & \OK & \NO & \OK
    & \NO & \NO & 4 \\
    
  SW19 & ASW0001ld7 & J020642.0-095157 & 0.2
    & IQ
    & \NO & \OK & \NO
    & \NO & \OK & 34 \\
    
  SW20 & ASW0002dx7 & J021221.1-105251 & 0.3
    & IQ
    & \OK & \OK & \OK
    & \NO & \OK & 6 \\
    
  SW21 & ASW0004m3x & J022533.3-053204 & 0.5
    & D
    & \OK & \NO & \NO
    & \NO & \OK & 2 \\
    
  SW22 & ASW0009ab8 & J022716.4-105602 & 0.4
    & D
    &  & \NO & \NO
    & \NO & \OK & 2 \\
    
  SW23 & ASW0003r61 & J023008.6-054038 & 0.6
    & ???
    & ? & ? & ?
    & ? & ? & 23 \\
    
  SW24 & ASW00050sk & J023315.2-042243 & 0.7
    & LQ
    & \NO & \OK & \NO
    & \OK & \OK & 2 \\
    
  SW25 & ASW00007mq & J090308.2-043252 & 
    & 
    &  &  & 
    &  &  &  \\
    
  SW26 & ASW0005ma2 & J135755.8+571722 & 0.8
    & D
    & \OK & \NO & \OK
    & \NO & \NO & 9 \\
    
  SW27 & ASW0006jh5 & J141432.9+534004 & 0.7
    & LQ
    & \NO & \NO & \NO
    & \NO & \OK & 10 \\
    
  SW28 & ASW0007xrs & J143055.9+572431 & 0.7
    & LQ
    & \NO & \OK & \NO
    & \OK & \OK & 3 \\
    
  SW29 & ASW0008qsm & J143838.1+572647 & 0.8
    & SQ
    & \NO & \OK & \OK
    & \OK & \OK & 4 \\
    
  SW30 & ASW0002p8y & J021057.9-084450 & 
    & 
    &  &  & 
    &  &  &  \\
    
  SW31 & ASW00021r0 & J021514.6-092440 & 0.7
    & LQ
    & \NO & \OK & \NO
    & \OK & \OK & 24 \\
    
  SW32 & ASW0004iye & J022359.8-083651 & 
    & 
    &  &  & 
    &  &  &  \\
    
  SW33 & ASW0003s0m & J022745.2-062518 & 0.6
    & D
    & \OK & \OK & \NO
    & \NO & \OK & 17 \\
    
  SW34 & ASW00051ld & J023453.5-093032 & 0.5
    & ???
    & ? & ? & ?
    & ? & ? & 10 \\
    
  SW35 & ASW0004wgd & J084833.2-044051 & 0.8
    & LQ
    & \NO & \OK & \NO
    & \OK & \OK & 5 \\
    
  SW36 & ASW000096t & J090248.4-010232 & 0.4
    & D
    & \OK & \OK & \NO
    & \NO & \OK & 9 \\
    
  SW37 & ASW00086xq & J143100.2+564603 & 
    & 
    &  &  & 
    &  &  &  \\
    
  SW38 & ASW0009cp0 & J143353.6+542310 & 0.8
    & LQ
    & \NO & \OK & \OK
    & \OK & \OK & 9 \\
    
  SW39 & ASW0005qiz & J220215.2+012124 & 
    & 
    &  &  & 
    &  &  &  \\
    
  SW40 & ASW0008wmr & J221306.1+014708 & 
    & 
    &  &  & 
    &  &  &  \\
    
  SW41 & ASW0008xbu & J221519.7+005758 & 0.4
    & IQ
    & \OK & \NO & \OK
    & \OK & \OK & 16 \\
    
  SW42 & ASW00096rm & J221716.5+015826 & 0.1
    & IQ
    & \OK & \OK & \NO
    & \OK & \NO & 5.0e+02 \\
    
  SW43 & ASW0001c3j & J020810.7-040220 & 1.0
    & IQ
    & \NO & \NO & \NO
    & \NO & \OK & 6 \\
    
  SW44 & ASW0002k40 & J021021.5-093415 & 0.4
    & ???
    & ? & ? & ?
    & ? & ? & 34 \\
    
  SW45 & ASW00024id & J021225.2-085211 & 0.8
    & R
    & \NO & \OK & \OK
    & \NO & \OK & 8 \\
    
  SW46 & ASW00024q6 & J021317.6-084819 & 0.5
    & D
    & \OK & \OK & \NO
    & \OK & \OK & 6 \\
    
  SW47 & ASW0003r6c & J022843.0-063316 & 0.5
    & D
    & \OK & \NO & \OK
    & \NO & \OK & 26 \\
    
  SW48 & ASW0000g95 & J090219.0-053923 & 
    & 
    &  &  & 
    &  &  &  \\
    
  SW49 & ASW00007ls & J090319.4-040146 & 
    & 
    &  &  & 
    &  &  &  \\
    
  SW50 & ASW00008a0 & J090333.2-005829 & 
    & 
    &  &  & 
    &  &  &  \\
    
  SW51 & ASW0006e0o & J135724.8+561614 & 
    & 
    &  &  & 
    &  &  &  \\
    
  SW52 & ASW0006a07 & J140027.9+541028 & 
    & 
    &  &  & 
    &  &  &  \\
    
  SW53 & ASW00070vl & J141518.9+513915 & 0.4
    & D
    & \OK & \NO & \OK
    & \NO & \OK & 15 \\
    
  SW54 & ASW0007sez & J142620.8+561356 & 0.5
    & R
    & \NO & \OK & \NO
    & \OK & \OK & 16 \\
    
  SW55 & ASW0007t5y & J142652.8+560001 & 
    & 
    &  &  & 
    &  &  &  \\
    
  SW56 & ASW0007pga & J142843.5+543713 & 0.4
    & D
    & \OK & \NO & \OK
    & \NO & \NO & 18 \\
    
  SW57 & ASW0008pag & J143631.5+571131 & 0.7
    & LQ
    & \NO & \OK & \NO
    & \NO & \NO & 64 \\
    
  SW58 & ASW0007iwp & J143651.6+530705 & 0.6
    & SQ
    & \NO & \NO & \OK
    & \OK & \OK & 19 \\
    
  SW59 & ASW00085cp & J143950.6+544606 & 
    & 
    &  &  & 
    &  &  &  \\
    


  \hline

\end{tabular}

\end{table*}

%%%%%%%%%%%%%%%%%%%%%%%%%%%%%%%%%%%%%%%%%%%%%%%%%%
%%%%%%%%%%%%%%%%%%%% REFERENCES %%%%%%%%%%%%%%%%%%
%%%%%%%%%%%%%%%%%%%%%%%%%%%%%%%%%%%%%%%%%%%%%%%%%%

% The best way to enter references is to use BibTeX:
\bibliographystyle{mnras}
\bibliography{bib/bibli} % if your bibtex file is called example.bib


%%%%%%%%%%%%%%%%%%%%%%%%%%%%%%%%%%%%%%%%%%%%%%%%%%
%%%%%%%%%%%%%%%%% APPENDICES %%%%%%%%%%%%%%%%%%%%%
%%%%%%%%%%%%%%%%%%%%%%%%%%%%%%%%%%%%%%%%%%%%%%%%%%

\section{TODO}
\listoftodos

\clearpage

\appendix

\input parts/app.tex

% Don't change these lines
\bsp	% typesetting comment
\label{lastpage}
\end{document}

