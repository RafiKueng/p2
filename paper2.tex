\documentclass[fleqn,usenatbib]{mnras}

%%% DEBUG SETTINGS - remove at the end %%%%%%%%%%%%%%%
\usepackage{silence} % silence error warinings
\WarningFilter{caption}{Unsupported document class} % caption doesn't know mnras..
\hbadness = 5000
\vbadness = 10000
% \usepackage{}

% MNRAS is set in Times font. If you don't have this installed (most LaTeX
% installations will be fine) or prefer the old Computer Modern fonts, comment
% out the following line
%\usepackage{newtxtext,newtxmath} % not yet supported on arxiv and uni computer
%\usepackage{txfonts}

% Depending on your LaTeX fonts installation, you might get better results with one of these:
%\usepackage{mathptmx}
%\usepackage{txfonts}

% Use vector fonts, so it zooms properly in on-screen viewing software
% Don't change these lines unless you know what you are doing
\usepackage[T1]{fontenc} % make sure font is supported
% \usepackage{ae,aecompl} %obsolete when using modern fonts
\usepackage[final]{microtype} % make sure font is supported
% and a suitable font
\usepackage{lmodern} % use a modern font with T1 support
%\usepackage{cm-super} % use a modern font with T1 support


%%%%% AUTHORS - PLACE YOUR OWN PACKAGES HERE %%%%%

% Only include extra packages if you really need them. Common packages are:
%\usepackage[dvipdfmx]{graphicx}	% Including figure files
\usepackage{graphicx} % Including figure files
%\usepackage[skip=0pt]{subcaption}
\usepackage{amsmath}	% Advanced maths commands
\usepackage{amssymb}	% Extra maths symbols
\usepackage{pifont}	% Extra maths symbols
%\usepackage{bmpsize}  % PS still needs this for correct bounding boxes?


% TODO stuff, comment out for final submit
\usepackage{todonotes} % as long as we are editing...
%\usepackage{showframe} % for debug
%\overfullrule=5pt      % show overfull boxes

% I disabled hyperlink in the cls und included it here because of bugs
\usepackage{hyperref}   % Hyperlinks
\hypersetup{colorlinks=true,linkcolor=blue,citecolor=blue,filecolor=blue,urlcolor=blue}

\newcommand{\inclfig}[2]{
  \centering
	\includegraphics[width=0.5\linewidth]{spaghetti/#2_input}%
	\includegraphics[width=0.5\linewidth]{spaghetti/#2_img1}
	\includegraphics[width=0.5\linewidth]{spaghetti/#2_img3_ipol}%
	\includegraphics[width=0.5\linewidth]{spaghetti/#2_img2}
	\includegraphics[width=\linewidth]{masses/#1_#2_mstel_vs_mtot}
}

\newcommand*{\rot}{\rotatebox{90}}
\newcommand*{\OK}{\ding{51}}
\newcommand*{\NO}{\ding{55}}

\newcommand{\lenstitle}[1]{\noindent\textbf{#1} --}
\newcommand{\params}[3]{(\(\Theta_\text{E}:#1\), $\varepsilon:#2$, $\alpha_\varepsilon:#3$ )}

\newcommand{\asw}[1]{ASW000#1}
\newcommand{\sw}[1]{SW~#1}
\newcommand{\model}[1]{SL model~#1}

\newcommand{\figref}[1]{Figure \ref{fig:#1}}
\newcommand{\Figref}[1]{Figure \ref{fig:#1}}

\title[Short title, max. 45 characters]{Model of lens candidates from
  Space Warps CFHTLS}

% The list of authors, and the short list which is used in the headers.
% If you need two or more lines of authors, add an extra line using \newauthor
\author[R. Kueng et al.]{
Rafael Kueng,$^{1}$\thanks{E-mail: rafael.kueng@uzh.ch}
Presenjit Saha,$^{1}$
Lucy Oswald$^{3}$
%and Fourth Author$^{4}$
\\
% List of institutions
$^{1}$Physik Institut, University of Zurich, Winterthurerstrasse XXX, Zurich 8057, Switzerland\\
$^{2}$Department, Institution, Street Address, City Postal Code, Country\\
$^{3}$Another Department, Different Institution, Street Address, City Postal Code, Country
%$^{4}$Another Department, Different Institution, Street Address, City Postal Code, Country
}

% These dates will be filled out by the publisher
% \date{Accepted XXX. Received YYY; in original form ZZZ}

% Enter the current year, for the copyright statements etc.
\pubyear{2016}

% Don't change these lines
\begin{document}
\label{firstpage}
\pagerange{\pageref{firstpage}--\pageref{lastpage}}
\maketitle

\begin{abstract}
This is a simple template for authors to write new MNRAS papers.
The abstract should briefly describe the aims, methods, and main results of the paper.
It should be a single paragraph not more than 250 words (200 words for Letters).
No references should appear in the abstract.
\end{abstract}

% Select between one and six entries from the list of approved keywords.
% Don't make up new ones.
\begin{keywords}
keyword1 -- keyword2 -- keyword3
\end{keywords}

\section{Introduction}

Space Warps \citep{2016MNRAS.455.1171M} found lens candidates
\citep{2016MNRAS.455.1191M}.

Modelling method \citep{2015MNRAS.447.2170K}.

\section{Methods}

\subsection{Stellar mass}

\subsection{Parameterisation of pixel models} \label{sec:parameter}
In order to fit the set of pixelated models to a single parameterised model, a program was written that took a parameterised function and subtracted from it the mean and the principle components of the data, which were calculated using classical Principle Component Analysis.
This created the residuals function.
The number of components defined as principle was varied to test how this affected the output, and it was found that using 5 principle components tended to give a reasonable approximation.
A masking function was added which selected only the data points that fell inside the image of the lens, and the principal components were clipped in order to keep the values inside the region of the ensemble of models.
Any value higher than the clip was set to be the clip value.
This was chosen to be 2.5 as, assuming that the data follows a Gaussian error distribution, almost all the values for the variance should lie between 2 and 3 standard deviations from the mean.
Minimising the residuals function produces the set of parameters that fit the parameterised function to the original pixelated ensemble most closely.
A least squares fit was used to perform this minimisation.


The parameterised model function was obtained from the gravitational potential of an isothermal ellipsoid mass distribution \cite{2001astro.ph..2341K}.
This model is frequently used to describe gravitational lenses as it tends to fit well with observations.
The isothermal ellipsoid model outputs three useful parameters: the radius of the Einstein ring, the ellipticity of the model and the angle of the ellipticity from the vertical, giving the orientation of the galaxy.
By applying this model to simulated lenses for which the values of these parameters were already known, it was possible to gain an estimate of the projected accuracy of the results, before applying the model to the candidate lensing galaxies.


\subsection{Additional / Improved Synthetic Image}

Volunteers often ask about better indicators to identify wether the modelling resulted in a ``good'' or ``bad'' model.
We developped a prototype of a better syntetic image than just the rendering of a \todo{insert currently used source profile} profile.
It operates on the composite input image.
The user has to mask the supposed lensed images.
The algorith then establishes a grid on the source plane and projects the masked pixels of the lense plane back to the source plane.
Multiple lens plane pixels fall onto one source plane pixel and get averaged.
The new synthetic image then reverses the mapping, coloring each pixel in the lens plane with the corresponding values of the source plane pixel.

\begin{figure}
  \includegraphics[width=\linewidth]{img/new_synth_img_detailed}
  \caption{
      Top-left: Orginal image, top-right: improved synthetic image for candidate \sw{5} (\asw{7k4r}) and model \model{12402};
      bottom from left to right:
        reconstructed source in color,
        intensity (grayscale),
        count of lens plane pixels per source plane pixel,
        residual of orginal image to improved syntetic image in grayscale
    }
  \label{fig:synthimg}
\end{figure}

\subsection{HiRes / Subsampling of central region}
In a previous paper \cite{2015MNRAS.447.2170K} we identified the problem of the tendency of the models to be too shallow.
We pinned down the problem to the central region and are confident to have it fixed with enabling sub pixel sampling in the central area.
The software allows for subsampling parameters of radius of pixels $r_\text{subs}$ and amount of subpixels per pixel $n_\text{subs}$.
\Figref{subsampling} shows the results of different settings for $r_\text{subs}$ and $n_\text{subs}$.
We can see that even the computationally least expensive settings of $r_\text{subs}=1$ and $n_\text{subs}=3$ leads to drastically improved profiles.

\begin{figure}
  \includegraphics[width=\linewidth]{hires/007022_kappa_encl}
  \caption{
    Circularly averaged mass profile for different subsampling of central region.
    Red line without subsampling, blue lines show subsampling with dash-dot pattern showing the mode.
    $\cdot / \cdot \cdot \cdot$: $r_\text{subs}=1$ inner pixels, each subdivided into $n_\text{subs}=3$ sub pixels.
    $\cdot / \cdot \cdot \cdot \cdot \cdot$: $r_\text{subs}=1$, $n_\text{subs}=5$.
    $\cdot \cdot / \cdot \cdot \cdot $: $r_\text{subs}=2$, $n_\text{subs}=3$.
    $\cdot \cdot \cdot / \cdot \cdot \cdot $: $r_\text{subs}=3$, $n_\text{subs}=3$.
    }
  \label{fig:subsampling}
\end{figure}

\section{Example systems}

\begin{figure}
  \inclfig{SW02}{ASW000619d_011489}
  \caption{SW02}
  \label{fig:SW02}
\end{figure}

\begin{figure}
  \inclfig{SW05}{ASW0007k4r_AJIBCHQ6EM}
  \caption{SW05}
  \label{fig:SW05}
\end{figure}

\begin{figure}
  \inclfig{SW09}{ASW0002asp_5EKMWWVJHL}
  \caption{SW09}
  \label{fig:SW09}
\end{figure}

\begin{figure}
  \inclfig{SW28}{ASW0007xrs_JHC3J2HYV7}
  \caption{SW28}
  \label{fig:SW28}
\end{figure}

\begin{figure}
  \inclfig{SW29}{ASW0008qsm_TOFS7JNGEK}
  \caption{SW29}
  \label{fig:SW29}
\end{figure}

Plausible

\begin{figure}
  \inclfig{SW58}{ASW0007iwp_4XBJWT3COV}
  \caption{SW58. topleft: user input; top right: resulting contour map of arrival time surface; mid left: synthetic image, renering a TODO{profile} source behind the modelled lens; mid right: mass profile; bottom: lensing mass, obtained from the model against estimate of stellar mass, obtained using photometric redshift assuming TODO(which population model?).}
  \label{fig:SW58}
\end{figure}

Uncertain

\begin{figure}
  \inclfig{SW36}{ASW000096t_7IPP7LWVOF}
\end{figure}

\begin{figure}
  \inclfig{SW19}{ASW0001ld7_OS3CYAKLRT}
\end{figure}


\begin{figure}
  \inclfig{SW57}{ASW0008pag_5SXGXQYY6V}
\end{figure}


\section{Summary of models}

Throu visual inspection of the most popular models generated for each candidate and data generated by those models we classified the candidates in four categories.
The data available is shown in \figref{SW58}.
\todo{better explain criterias?}
The following sections list all the models for all the categories, each with a short explanation of the reasons.
Additionally, selected models are presented in figures for demonstration.

\subsection{plausible}

The plausible candidates fail in one category, but are quite convincing in all the others.
One particular reason for a candidate to end up in this category is that is has extended arcs, what is hard to modell with the current version of the program.
Thats why we recomend them for a following up study, for example for taking spectras or other follow up obersatories and for other modelling programs for a try.


\subsection{unclear}

This categy contains the ones that are hard to say.
They might look like lenses visually, but the modelling failed or prediced additional images.


\subsection{convincing}

The canidates that looked the most promising and left little doubt about being lenses were grouped into the category ``convincing''.
In Figures \ref{fig:SW02} to \ref{fig:SW29} we show all the convincing candidates.
Following we list the parameters, obtained applying the technique described in section~\ref{sec:parameter}

\lenstitle{SW02 (ASW000619d)}
Nice arc with counter image;
additional mass at 12 o'clock doesn't seem to influence the image.
\params{1.04}{0.32}{-5.18}

\lenstitle{SW05 (ASW0007k4r)}
The model predicts very steep central mass profile, and thus a very high lensing mass;
looks otherwise very convincing.
\params{3.80}{0.15}{52.50}
  
\lenstitle{SW09 (ASW0002asp)}
\params{1.12}{0.00}{88.39}
  
\lenstitle{SW28 (ASW0007xrs)}
Faint, but clearly visible arc and counter image;
but model sfits nicely
\params{1.04}{0.33}{66.63}
 
\lenstitle{SW29 (ASW0008qsm)}
Faint close arc, but reconstruion works very well;
parameterisation of ellypticity fails.
\params{0.73}{0.00}{249.80}


\subsection{doubtful}

\begin{figure}
  \inclfig{SW42}{ASW00096rm_4Q3YCEWGLN}
  \caption{outlier because very odd mass ratio}
  \label{fig:SW42}
\end{figure}







\subsection{missing}

Finally this category lists all the candidates, that miss redshift measurements of the lens and thus further analysis is difficult.

\lenstitle{SW03 (ASW0006mea)}
\lenstitle{SW14 (ASW0004xjk)}
\lenstitle{SW25 (ASW00007mq)}
\lenstitle{SW30 (ASW0002p8y)}
\lenstitle{SW37 (ASW00086xq)}
\lenstitle{SW39 (ASW0005qiz)}
\lenstitle{SW40 (ASW0008wmr)}
\lenstitle{SW48 (ASW0000g95)}
\lenstitle{SW49 (ASW00007ls)}
\lenstitle{SW50 (ASW00008a0)}
\lenstitle{SW55 (ASW0007t5y)}
\lenstitle{SW59 (ASW00085cp)}


\subsection{Table}

\begin{table*}
  \caption{Categorisation of SW models}
  \label{tab:models}
  
\begin{tabular}{c c c | c | c c c | c c c}
  \hline
  SWID & ASW id & model id
    & \rot{\shortstack[l]{spaghetti\\type}}
    
    & \multicolumn{1}{|l|}{\rot{\shortstack[l]{blending\\into arc}}}
    & \rot{\shortstack[l]{unseen\\counter\\image}}
    & \rot{\shortstack[l]{lensing\\pertubation}}
    
    & \rot{\shortstack[l]{img reconstr\\reasonable}}
    & \rot{\shortstack[l]{mass reconstr\\reasonable}}
    & \rot{M/L ratio}
  \\ \hline
 
% this is an example entry
%  SW99 & ASW000XXXX & 012345 & 0+0
%    & \NO & \NO & \OK
%    & \OK & \NO & \NO \\

  SW01 & ASW0004dv8 & 
    & X+X
    & \NO & \NO & \NO
    & \NO & \NO & \NO \\
    
  SW02 & ASW000619d & 011489
    & X+X
    & \NO & \NO & \NO
    & \NO & \NO & \NO \\
    
  SW03 & ASW0006mea & 
    & X+X
    & \NO & \NO & \NO
    & \NO & \NO & \NO \\
    
  SW04 & ASW0009cjs & NJ5CC5YJAQ
    & X+X
    & \NO & \NO & \NO
    & \NO & \NO & \NO \\
    
  SW05 & ASW0007k4r & AJIBCHQ6EM
    & 1+2+1
    & \NO & \NO & \NO
    & \OK & \OK & $\sim100$ \\
    
  SW06 & ASW0008swn & BCY2NOUSLK
    & X+X
    & \NO & \NO & \NO
    & \NO & \NO & \NO \\
    
  SW07 & ASW0007e08 & 
    & X+X
    & \NO & \NO & \NO
    & \NO & \NO & \NO \\
    
  SW08 & ASW00099ed & HISGRAIZL2
    & X+X
    & \NO & \NO & \NO
    & \NO & \NO & \NO \\
    
  SW09 & ASW0002asp & 5EKMWWVJHL
    & X+X
    & \NO & \NO & \NO
    & \NO & \NO & \NO \\
    
  SW10 & ASW0002bmc & VQYCYNONVW
    & X+X
    & \NO & \NO & \NO
    & \NO & \NO & \NO \\
    
  SW11 & ASW0002qtn & 3TUJKHGED4
    & X+X
    & \NO & \NO & \NO
    & \NO & \NO & \NO \\
    
  SW12 & ASW0003wsu & 012712
    & X+X
    & \NO & \NO & \NO
    & \NO & \NO & \NO \\
    
  SW13 & ASW00047ae & TGTIIF7HCV
    & X+X
    & \NO & \NO & \NO
    & \NO & \NO & \NO \\
    
  SW14 & ASW0004xjk & 
    & X+X
    & \NO & \NO & \NO
    & \NO & \NO & \NO \\
    
  SW15 & ASW0004nan & QUOGDU2NN6
    & X+X
    & \NO & \NO & \NO
    & \NO & \NO & \NO \\
    
  SW16 & ASW0009bp2 & 013421
    & X+X
    & \NO & \NO & \NO
    & \NO & \NO & \NO \\
    
  SW17 & ASW0005rnb & AAKHMTYTMS
    & X+X
    & \NO & \NO & \NO
    & \NO & \NO & \NO \\
    
  SW18 & ASW0007hu2 & D4UQI6M3ZU
    & X+X
    & \NO & \NO & \NO
    & \NO & \NO & \NO \\
    
  SW19 & ASW0001ld7 & OS3CYAKLRT
    & X+X
    & \NO & \NO & \NO
    & \NO & \NO & \NO \\
    
  SW20 & ASW0002dx7 & 3NYJG67KRT
    & X+X
    & \NO & \NO & \NO
    & \NO & \NO & \NO \\
    
  SW21 & ASW0004m3x & QZROE23AUH
    & X+X
    & \NO & \NO & \NO
    & \NO & \NO & \NO \\
    
  SW22 & ASW0009ab8 & TGM4U2TZBS
    & X+X
    & \NO & \NO & \NO
    & \NO & \NO & \NO \\
    
  SW23 & ASW0003r61 & 002481
    & X+X
    & \NO & \NO & \NO
    & \NO & \NO & \NO \\
    
  SW24 & ASW00050sk & 013406
    & X+X
    & \NO & \NO & \NO
    & \NO & \NO & \NO \\
    
  SW25 & ASW00007mq & 
    & X+X
    & \NO & \NO & \NO
    & \NO & \NO & \NO \\
    
  SW26 & ASW0005ma2 & 5ZZKUM3SWL
    & X+X
    & \NO & \NO & \NO
    & \NO & \NO & \NO \\
    
  SW27 & ASW0006jh5 & 5URN3BQFSV
    & X+X
    & \NO & \NO & \NO
    & \NO & \NO & \NO \\
    
  SW28 & ASW0007xrs & JHC3J2HYV7
    & X+X
    & \NO & \NO & \NO
    & \NO & \NO & \NO \\
    
  SW29 & ASW0008qsm & TOFS7JNGEK
    & X+X
    & \NO & \NO & \NO
    & \NO & \NO & \NO \\
    
  SW30 & ASW0002p8y & 
    & X+X
    & \NO & \NO & \NO
    & \NO & \NO & \NO \\
    
  SW31 & ASW00021r0 & SYTNGELH3Q
    & X+X
    & \NO & \NO & \NO
    & \NO & \NO & \NO \\
    
  SW32 & ASW0004iye & 
    & X+X
    & \NO & \NO & \NO
    & \NO & \NO & \NO \\
    
  SW33 & ASW0003s0m & ECXCIRBDUJ
    & X+X
    & \NO & \NO & \NO
    & \NO & \NO & \NO \\
    
  SW34 & ASW00051ld & 000291
    & X+X
    & \NO & \NO & \NO
    & \NO & \NO & \NO \\
    
  SW35 & ASW0004wgd & VWJ2LNN3VZ
    & X+X
    & \NO & \NO & \NO
    & \NO & \NO & \NO \\
    
  SW36 & ASW000096t & 7IPP7LWVOF
    & X+X
    & \NO & \NO & \NO
    & \NO & \NO & \NO \\
    
  SW37 & ASW00086xq & 
    & X+X
    & \NO & \NO & \NO
    & \NO & \NO & \NO \\
    
  SW38 & ASW0009cp0 & Z6IFI4SLLM
    & X+X
    & \NO & \NO & \NO
    & \NO & \NO & \NO \\
    
  SW39 & ASW0005qiz & 
    & X+X
    & \NO & \NO & \NO
    & \NO & \NO & \NO \\
    
  SW40 & ASW0008wmr & 
    & X+X
    & \NO & \NO & \NO
    & \NO & \NO & \NO \\
    
  SW41 & ASW0008xbu & BFXRMIQEAT
    & X+X
    & \NO & \NO & \NO
    & \NO & \NO & \NO \\
    
  SW42 & ASW00096rm & 4Q3YCEWGLN
    & X+X
    & \NO & \NO & \NO
    & \NO & \NO & \NO \\
    
  SW43 & ASW0001c3j & 5R6UYQZUTI
    & X+X
    & \NO & \NO & \NO
    & \NO & \NO & \NO \\
    
  SW44 & ASW0002k40 & 000899
    & X+X
    & \NO & \NO & \NO
    & \NO & \NO & \NO \\
    
  SW45 & ASW00024id & TSKKYHD3CB
    & X+X
    & \NO & \NO & \NO
    & \NO & \NO & \NO \\
    
  SW46 & ASW00024q6 & 012523
    & X+X
    & \NO & \NO & \NO
    & \NO & \NO & \NO \\
    
  SW47 & ASW0003r6c & 4HC3CREEAD
    & X+X
    & \NO & \NO & \NO
    & \NO & \NO & \NO \\
    
  SW48 & ASW0000g95 & 
    & X+X
    & \NO & \NO & \NO
    & \NO & \NO & \NO \\
    
  SW49 & ASW00007ls & 
    & X+X
    & \NO & \NO & \NO
    & \NO & \NO & \NO \\
    
  SW50 & ASW00008a0 & 
    & X+X
    & \NO & \NO & \NO
    & \NO & \NO & \NO \\
    
  SW51 & ASW0006e0o & 
    & X+X
    & \NO & \NO & \NO
    & \NO & \NO & \NO \\
    
  SW52 & ASW0006a07 & 
    & X+X
    & \NO & \NO & \NO
    & \NO & \NO & \NO \\
    
  SW53 & ASW00070vl & BPAV4GVOPP
    & X+X
    & \NO & \NO & \NO
    & \NO & \NO & \NO \\
    
  SW54 & ASW0007sez & SI4ELBAKL2
    & X+X
    & \NO & \NO & \NO
    & \NO & \NO & \NO \\
    
  SW55 & ASW0007t5y & 
    & X+X
    & \NO & \NO & \NO
    & \NO & \NO & \NO \\
    
  SW56 & ASW0007pga & VHV6RQYYKZ
    & X+X
    & \NO & \NO & \NO
    & \NO & \NO & \NO \\
    
  SW57 & ASW0008pag & 5SXGXQYY6V
    & X+X
    & \NO & \NO & \NO
    & \NO & \NO & \NO \\
    
  SW58 & ASW0007iwp & 4XBJWT3COV
    & X+X
    & \NO & \NO & \NO
    & \NO & \NO & \NO \\
    
  SW59 & ASW00085cp & 
    & X+X
    & \NO & \NO & \NO
    & \NO & \NO & \NO \\
    


  \hline

\end{tabular}

\end{table*}



\section{Discussions}





%%%%%%%%%%%%%%%%%%%%%%%%%%%%%%%%%%%%%%%%%%%%%%%%%%
%%%%%%%%%%%%%%%%%%%% REFERENCES %%%%%%%%%%%%%%%%%%
%%%%%%%%%%%%%%%%%%%%%%%%%%%%%%%%%%%%%%%%%%%%%%%%%%

% The best way to enter references is to use BibTeX:
\bibliographystyle{mnras}
\bibliography{bib/bibli} % if your bibtex file is called example.bib


%%%%%%%%%%%%%%%%%%%%%%%%%%%%%%%%%%%%%%%%%%%%%%%%%%
%%%%%%%%%%%%%%%%% APPENDICES %%%%%%%%%%%%%%%%%%%%%
%%%%%%%%%%%%%%%%%%%%%%%%%%%%%%%%%%%%%%%%%%%%%%%%%%

\appendix

\section{Some extra material}

If you want to present additional material which would interrupt the flow of the main paper,
it can be placed in an Appendix which appears after the list of references.

%%%%%%%%%%%%%%%%%%%%%%%%%%%%%%%%%%%%%%%%%%%%%%%%%%

\clearpage

\section{TODO}
\listoftodos

\todo{REMOVE TODOS at the end}



% Don't change these lines
\bsp	% typesetting comment
\label{lastpage}
\end{document}

