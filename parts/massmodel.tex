\section{Mass models}\label{sec:massmodels}

A SpaghettiLens model consists of a statistical ensemble of free-form
maps of the sky-projected mass distribution responsible for lensing.
  Photometric lens
redshifts are used; source redshifts are set to $z=2$ unless an
unambiguous photometric redshift is available.  The models can,
however, be trivially rescaled to use better redshift values, as and
when they become available.

A spaghetti diagram is read by a server-side numerical engine
\citep[GLASS, developed by][]{2014MNRAS.445.2181C} which then returns
a statistical ensemble of mass maps.  The mass maps are made up of
mass tiles and are free-form, except that they are required to be
concentrated around the identified lens centre.  They are also
required to reproduce the given image locations, parities and time
ordering exactly.  Graphical representations of the mass map and
arrival-time surface are returned to the user for review.  The user
can post these results on a forum, or discard them and try again.
Volunteers can start the modelling process afresh, or they can take an
existing model from the forum and modify its input spaghetti diagram
or its accompanying options, and thus obtain a revised model.

Alongside is a synthetic image produced by modelling.  In it, the
fitted lensed images are shown in colour, while the lensing galaxy and
extraneous objects have been reduced to grayscale.
(Figure~\figref{synthimg} in \S~\ref{subsec:sourcefit} shows a
synthetic image from another model of the same system.) These two
panels are qualitative and display no units, and moreover, the mutual
alignment of the two panels is only approximate.

Figures~\figref{SW58} and \figref{SW19} appear plausible lens
candidates.  Their morphology is similar to SW28, but the saddle-point
counterimage is not visible.  We consider these lenses plausible but
less convincing.

Figures~\figref{SW36} and \figref{SW57} are cases where modelling
failed.

\begin{enumerate}
\item The middle row shows contour maps from the model.  At middle
  left, we have the arrival-time surface.
\item At middle right we have the mass distribution in the usual
  dimensionless form $\kappa$.  Again, these two panels are mainly
  qualitative: both panels are spatially registered and centred on the
  density-peak of the lens, but no scales have been included.
\item The base row supplies spatial and mass scales.  The left panel
  shows the circularly-averaged $\kappa$ of the model ensemble, with
  increasing radius (in arcsec).  The four short vertical lines
  correspond to the minima and saddle points marked in the upper-left
  panel.  The effective Einstein radius \ER is also shown.  As the radius
  scale indicates SW05 is comparatively large lens on the
  sky.
\item 
\end{enumerate}

