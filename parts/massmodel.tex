\section{Mass models}\label{sec:massmodels}

Once a spaghetti diagram has been drawn on a web browser, it is
forwarded to a server-side numerical framework, which searches for
mass maps consistent with the given image locations, parities and time
ordering.  The mass maps are made up of mass tiles and are free-form,
but are required to be concentrated around the identified lens centre
\citep[see][for the precise formulation of the search
  problem.]{2014MNRAS.445.2181C} Assuming such mass distributions can
be found (in practise, usually the case) a statistical ensemble of
two-dimensional mass maps is returned.  This ensemble, along with
derived quantities and uncertainties, makes up one SpaghettiLens
model.

The mass distribution will naturally depend on the lens and source
redshifts, which are unknown when a lens candidate is first
identified.  But this is not a problem, because a model can be
trivially rescaled to use better redshift values, as and when they
become available.  This paper uses pipeline photometric redshifts for
the candidate lensing galaxies.  Source redshifts are set to $z=2$
unless an unambiguous photometric redshift is available.

Figure~\ref{fig:arriv}

Figure~\ref{fig:synth}
New method, see also Figure~\figref{synthimg} in
Appendix~\ref{subsec:sourcefit}.

Figure~\ref{fig:kappa}

Figure~\ref{fig:encl}. Appendix~\ref{subsec:hires} describes
improvements made since our earlier work \citep{2015MNRAS.447.2170K}



\endinput



  The ensemble average
can, however, be a useful representative of the whole ensemble.  The
user interface of SpaghettiLens returns graphical representations of
the ensemble-average mass map and some derived quantities to the
modeller for review.  The modeller can post these results on a forum,
or discard them and try again.  Other volunteers can start the
modelling process afresh, or they can take an existing model from the
forum and modify its input spaghetti diagram or its accompanying
options, and thus obtain a revised model.



Alongside is a synthetic image produced by modelling.  In it, the
fitted lensed images are shown in colour, while the lensing galaxy and
extraneous objects have been reduced to grayscale.
) These two
panels are qualitative and display no units, and moreover, the mutual
alignment of the two panels is only approximate.

Figures~\figref{SW58} and \figref{SW19} appear plausible lens
candidates.  Their morphology is similar to SW28, but the saddle-point
counterimage is not visible.  We consider these lenses plausible but
less convincing.

Figures~\figref{SW36} and \figref{SW57} are cases where modelling
failed.

\begin{enumerate}
\item The middle row shows contour maps from the model.  At middle
  left, we have the arrival-time surface.
\item At middle right we have the mass distribution in the usual
  dimensionless form $\kappa$.  Again, these two panels are mainly
  qualitative: both panels are spatially registered and centred on the
  density-peak of the lens, but no scales have been included.
\item The base row supplies spatial and mass scales.  The left panel

\item 
\end{enumerate}

