\section{Introduction}

Light deflection at the rim of the Sun is famously $1.75''$.  The
deflection angle can also be expressed in terms of the escape velocity
$2v_{\rm esc}^2/c^2$, with $v_{\rm esc}$ at the solar surface being
$\simeq620\rm\,km/s$.  The Galactic escape velocity at the Sun's
location is similar: $\approx500\rm\,km/s$, typical of massive
galaxies.  Thus, the outer regions of massive galaxies have lensing
deflections comparable to that at the rim of the Sun.  Yet these
quantitatively similar deflections have qualitatively different
consequences.  Light deflection by the Sun must be accounted for in
modern astrometry \citep[see e.g.,][]{2015CQGra..32p5008C} but is not
itself a physics probe in the way it was a century ago.  With distant
galaxies, however, light bending by $\sim1''$ introduces two new
phenomena.  First, for galaxies distant enough that their apparent
size is comparable to the bending angle, light deflection can cause
multiple images of background sources, or strong lensing.  Second, the
gravitational field is dominated by dark matter, hence strong lensing
becomes a probe of dark matter in galaxies.

With no unambiguous dark-matter particle detections so far, dark
matter has studied only through its indirect consequences on galactic
and larger scales.  In most work, dark matter is taken to be a
collisionless non-relativistic fluid (cold dark matter or CDM); in
recent years, the simulation by \cite{2005Natur.435..629S} has been
particularly influential in studying the formation of CDM structures.
Other scenarios have also been considered, such as a cold condensing
boson fluid \citep{2016ApJ...818...89S}.  There is a general
consensus, however, that the origins of galaxy lie in the
gravitational collapse, fragmentation, and mergers of dark-matter
clumps, into which fell gas, cooling through radiative processes to
form dense clouds and eventually stars.  There is, however, much
debate about the details, of which \cite{2012RAA....12..917S} provide
a nice summary.

Strong-lensing galaxies are a useful source of information on the
mutual dynamics and dark matter and gas in galaxies.  The topic has
been explored in several studies
\citep{2009ApJ...703L..51K,2011ApJ...740...97L,2012MNRAS.424..104L,
  2016MNRAS.459.3677L,2016MNRAS.456..870B} but it is desirable to
enlarge the samples from tens of lensing galaxies to thousands.  Doing
so requires finding more lenses, of course, but also modelling their
masses.

Recent searches through the CFHTLS \citep{2012SPIE.8448E..0MC} using
arc-finders
\citep{2012ApJ...749...38M,2014A&A...567A.111M,2014ApJ...785..144G} by
machine learning \citep{2016arXiv160504309P} and by visual inspection
by a citizen-science volunteers
\citep[Space~Warps][]{2016MNRAS.455.1191M} have between them
discovered an average of four lenses per square degree.  So one can be
optimistic about finding many thousands of lenses in the next
generation of wide-field surveys.  The expected flood of new lens
discoveries will need a similarly huge modelling effort to reconstruct
their mass distributions.  To prepare for the challenge of
massive-sample lens modelling, \cite{2015MNRAS.447.2170K} developed a
new modelling strategy, implemented as the SpaghettiLens system.  The
idea is to collaborate with experienced members of the citizen-science
community, who have already participated in lens discovery through
Space~Warps, as well as several other projects involving astronomical
data.  In this paper, we present SpaghettiLens models of lens-candidates
discovered through Space~Warps.

Purpose is twofold.  First, see if plausible mass models can produce
the overall configuration.  Second, interpretation as lensing
galaxies.  Introduce some diagnostics.

Each lens candidates in practice gets modelled several times in a
collaborative refinement process.\footnote{See ``Collaborative
  gravitational lens modelling\dots'' in {\tt
    http://letters.zooniverse.org} especially the model tree.}  This
representing a consensus among modellers, as to the best that could be
achieved with the available data and software.\footnote{This is in
  contrast to the main Space~Warps project for discovering lenses, in
  which volunteers in a crowd of $\gtrsim10^4$ make independent
  contributions.  Each person is presented with a random selection of
  survey-patches and invited to (in effect) vote on each.  The system
  estimates each volunteer's skill level according to test-patches
  interspersed with the real data, and weights their votes accordingly
  \citep{2016MNRAS.455.1171M}.  There is an active forum for
  volunteers, but since everyone is seeing different data samples with
  minimal overlap, the forum has little if any influence on votes.}

\begin{itemize}
\item Section~\ref{sec:morph} discusses nine of the systems, including
  six systems we found to be particularly interesting.  Morphology
  from Figure~\ref{fig:markedup}.
\begin{itemize}
\item The image morphology concisely described: D for double, and
  quads in sub-categorised in four ways
  \citep[cf.][]{2003AJ....125.2769S} as LQ for long-axis quads (as in
  SW28), SQ for short-axis quads (as in Figure~SW02,
  SW09, SW29, SW42), IQ for inclined quads
  (as in Figure~SW05) and CQ for very symmetric core quads.
\item Whether the images are unblended.  Distinct unblended images (as
  in Figures~SW05 and SW42) are an advantage in
  modelling, but not essential.
\item Whether all images are discernable.  The topography of an
  arrival-time surface, as encoded by a spaghetti diagram, may require
  more images than are visible.  For example, in Figure~SW58,
  the modeller has put in a conjectural saddle point near the lensing
  galaxy.
\item Whether the lens is fairly isolated.
\end{itemize}
\item Section~\ref{sec:massmodels}
\begin{itemize}
\item Whether the mass map is reasonable. Figure~\ref{fig:kappa}.
\item Whether the arrival-time surface and synthetic image are
  plausible.  For SW36 the model implies extra images or a long arc,
  which are not seen.  Figures~\ref{fig:arriv} and \ref{fig:synth}.
\end{itemize}
\item Section~\ref{sec:stellar-mass} explains how we compare the lensing
mass with the mass in stars in the lensing galaxy.  We estimate the
stellar mass by comparing galaxy magnitudes from the CFHTLS pipeline
with the well-known stellar-population models of
\cite{2003MNRAS.344.1000B}.  We then extrapolate the stellar masses to
a halo mass using the abundance-matching prescription of
\cite{2010ApJ...710..903M}.  Naturally, the lensing mass must be more
than the stellar mass but no more than the total halo mass.  We then
introduce what we call a halo index $\haloindex$ which gives an idea
of how the lensing mass compares with these two bounds.
Figure~\ref{fig:stelmass}.
\end{itemize}







In this paper we report the most recent model for each lens, as










Section~\ref{sec:summary} builds Table~\ref{tab:models}.
The online supplement gives results for all the modelled systems.

Appendices devoted to various technical issues relating to modelling.
\cite{2015MNRAS.447.2170K} tested the system on simulated lenses and
identified some areas for improvement.
In \S~\ref{subsec:sourcefit} we introduce fitting of the brightness
profiles of the source.  This feature has not yet been included in
SpaghettiLens, but has been carried out in post-processing for a few
especially interesting candidates.  In \S~\ref{subsec:hires} we show
that making mass maps fine-grained in the central region relieves a
tendency in the earlier work for mass to be too shallow. Then in
\S~\ref{subsec:parameter} we consider the possibility of fitting a
parametric lens model to the model ensemble; so far we have only been
successful at extracting an Einstein radius.
