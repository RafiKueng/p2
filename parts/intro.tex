\section{Introduction}

By coincidence, the typical escape velocity of massive galaxies is
such that $v_{\rm esc}^2/c^2$ is comparable to the apparent sizes of a
galaxies at cosmological distances.  This coincidence is fortunate,
because it makes the lensing deflection angle (which is $2v_{\rm
  esc}^2/c^2$) of distant galaxies comparable to their size on the
sky, and as a result, strong lensing by galaxies tends to produce
images that probe the dark halos of those galaxies.  This is
important, because while there is a general consensus that basic
mechanism of galaxy formation involve gravitational collapse,
fragmentation, and mergers of dark-matter clumps, into which gas fell,
cooling through radiative processes to form dense clouds and
eventually stars, there is much debate about the details \citep[for a
  summary, see][]{2012RAA....12..917S}.  In particular, the nature of
dark matter remains mysterious: most researchers take it to be a
collisionless non-relativistic fluid (cold dark matter or CDM) readily
studied by simulations \citep[for example, the millenium simulation
  by][which has been particularly influential]{2005Natur.435..629S}.
But other scenarios have also been considered, such as a cold
condensing boson fluid \citep{2016ApJ...818...89S}, or dark matter
particles with internal degrees of freedom
\citep{2010MNRAS.405...77S}, or not matter at all but a modification
of gravity \citep{2016PhRvL.117t1101M}.

All this motivates using galaxy lenses to study the mutual dynamics of
dark matter and gas in galaxies.  Several studies in recent years have
done so
\citep{2009ApJ...703L..51K,2011ApJ...740...97L,2012MNRAS.424..104L,
  2016MNRAS.459.3677L,2016MNRAS.456..870B} but it is desirable to
enlarge the samples from tens of lensing galaxies to thousands.  Doing
so requires both finding more lenses and also modelling their masses.
Recent searches through the CFHTLS \citep{2012SPIE.8448E..0MC} using
arc-finders
\citep{2012ApJ...749...38M,2014A&A...567A.111M,2014ApJ...785..144G} by
machine learning \citep{2016A&A...592A..75P} and by visual inspection
by citizen-science volunteers in Space~Warps
\citep{2016MNRAS.455.1191M} have, between them, discovered an average
of four lenses per square degree.  So one can be optimistic about
finding many thousands of lenses in the next generation of wide-field
surveys, from the LSST in optical and the SKA in radio on the ground,
and Euclid and WFIRST in orbit.

The expected flood of new lens discoveries will need a similarly huge
modelling effort to reconstruct their mass distributions.  To prepare
for the challenge of massive-sample lens modelling,
\cite{2015MNRAS.447.2170K} developed a new modelling strategy,
implemented as the SpaghettiLens system.  The idea is to collaborate
with experienced members of the citizen-science community, who have
already participated in lens discovery through Space~Warps, as well as
several other projects involving astronomical data.  The system was
tested on a sample of simulated lenses from Space~Warps.

This paper continues by applying SpaghettiLens to lens-candidates
discovered through Space~Warps.  We present results from modelling of
42 of the 59 lens candidates reported by \cite{2016MNRAS.455.1191M}.
Each lens candidate was modelled in a collaborative refinement
process, where anyone interested can create a new model or modify an
existing model to try and make it better.\footnote{This is in contrast
  to the main Space~Warps project for discovering lenses, where
  volunteers in a crowd of $\gtrsim10^4$ make independent
  contributions.  Each person is presented with a random selection of
  survey-patches and invited to (in effect) vote on each.  The system
  estimates each volunteer's skill level according to test-patches
  interspersed with the real data, and weights their votes accordingly
  \citep{2016MNRAS.455.1171M}.  There is an active forum for
  volunteers, but since everyone is seeing different data samples with
  minimal overlap, the forum has little if any influence on votes.}
The result model represents a consensus among contributors, as to the
best that could be achieved with the available data and software.

We characterise each model with seven diagnostics, whose purpose is to
help identify which systems are most probably lenses, and which ones
are likely to be most rewarding for future follow-up observations.
The diagnostics are as follows.

\begin{itemize}
\item First we have diagnostics based on morphology of the
  system.
  Section~\ref{sec:morph} and Figure~\ref{fig:splinput} explain.
\begin{itemize}
\item Whether the images are unblended.  Distinct unblended images are
  an advantage in modelling, but not essential.
\item Whether all images are discernible.  The topography of an
  arrival-time surface, as encoded by a spaghetti diagram, may require
  more images than are visible, in which case the modeller has to
  insert conjectural image positions.
\item Whether the lens is fairly isolated.
\item The image morphology concisely described: double or quads,
  further sub-categorised to indicate the elongation direction of the
  lensing mass.
\end{itemize}
\item Second we have mass models, covered in Section~\ref{sec:massmodels}.
\begin{itemize}
\item Whether the mass map is reasonable. Figure~\ref{fig:kappa}.
\item Whether the arrival-time surface and synthetic image are
  plausible.  In particular, additional images are implied in regions
  where they are not observed signal an unsatisfactory model.
  Figures~\ref{fig:arriv} and \ref{fig:synth} and \ref{fig:encl}.
\end{itemize}
\item Third, whether the implied lensing mass is plausible, given the
  photometric data of the lensing galaxy.  Section~\ref{sec:stellar-mass}
  explains how we compare the lensing mass with the mass in stars in
  the lensing galaxy.  We estimate the stellar mass by comparing
  galaxy magnitudes from the CFHTLS pipeline with the well-known
  stellar-population models of \cite{2003MNRAS.344.1000B}.  We then
  extrapolate the stellar mass to a halo mass using the
  abundance-matching prescription of \cite{2010ApJ...710..903M}.
  Naturally, the lensing mass must be more than the stellar mass but
  no more than the total halo mass.  We then introduce what we call a
  halo index ($\haloindex$) which gives an idea of how the lensing mass
  compares with these two contraints.  Figure~\ref{fig:stelmass}.
\end{itemize}

Section~\ref{sec:summary} summarises and tabulates the diagnostics in
Table~\ref{tab:models}.  Interpretation of the results is preliminary,
because the systems are candidates at this stage, not secure lenses.
Moreover the candidate-lens redshifts have large uncertainties, while
the candidate-source redshifts can only be guessed at present.
Nevertheless it is interesting to see what trends we can observe with
the already-available data.

There are three appendices devoted to various technical issues
relating to modelling.  \cite{2015MNRAS.447.2170K} tested the system
on simulated lenses and identified some areas for improvement.  In
\S~\ref{subsec:sourcefit} we introduce fitting of the brightness
profiles of the source.  This feature has not yet been included in
SpaghettiLens, but has been carried out in post-processing for a few
especially interesting candidates.  In \S~\ref{subsec:hires} we show
that making mass maps fine-grained in the central region relieves a
tendency in the earlier work for mass to be too shallow. Then in
\S~\ref{subsec:parameter} we consider the possibility of fitting a
parametric lens model to the model ensemble; so far we have only been
successful at extracting an Einstein radius.

The online supplement gives results for all the modelled systems generated
for all the lensing candidates.

