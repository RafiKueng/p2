
%%%%%%%%%%%%%%%%%%%%%%%%%%%%%%%%%%%%%%%%%%%%%%%%%%%%%%%%%%%%%%%%%%%%%%%
\section{Summary and conclusions}\label{sec:summary}


We report on a first set of mass distributions and follow-up
diagnostics for the {\SW} lens candidates created with a novel
approach that aims to be scalable by {\sl orders of magnitude} to
prepare for the many thousands of lenses the next generation of wide
field surveys will yield (e.g. Euclid, WFIRST).

Over the past few years, the way of discovering lenses has changed
with the introduction of machine learning and citizen science methods,
combined with the coverage of large areas of the sky by modern
surveys.  The way lensing mass models are constructed also needs to
change, in order to be prepared for the increasing influx of lens
candidates.  The work in this paper represents a hybrid approach
between the classical style -- where a small team of experts invest
many hours into the creation of a single model -- and a citizen
science project -- where a crowd of amateur volunteers make
independent contributions.  The authors of this paper are a
collaboration of professionals and experienced citizen-science
volunteers, aiming to create early-stage lens models as soon as a lens
candidate is found.

To assist volunteers in constraining lensing models, we introduce a
set of diagnostics that help asses the validity of the models. At a
later stage, we encourage modellers to apply those diagnostics as
feedback on the plausibility of their assumptions, and to suggest
additional diagnostics.

The diagnostics (i) -- (iv) (see Section.~\ref{sec:candidates_models})
turned out to be useful measures of the difficulty in modelling a
system, but they did not constitute necessary conditions for a
promising model.  They can help select systems to introduce novice
volunteers to the modelling process. In contrast, diagnostics (v) and
(vi) can be considered as necessary criteria for a good model.
Volunteers employed those ones to evaluate their models, and turned out
to be easy enough to grasp by new volunteers.  The halo-matching
index diagnostic (vii, $\haloindex$), is an interesting criterion that
might be useful for the modellers, but needs further investigation.


Table~\ref{tab:models} is a summary of our results.  It characterises
each modelled system with seven diagnostics, indicating (a)~the image
morphology and how clear or indistinct it is, (b)~whether the mass map
and synthetic lensed image appear to be plausible, and (c)~how the
model mass compares with the estimated stellar and full-halo masses.
Missing entries are due unavailable imaging data, whereas 
missing rows are due to models that were not created for this particular
candidate.

The trend in Fig.~\ref{fig:stelmass}, where higher-mass galaxies get
progressively more dark-matter dominated, is expected
\citep[see, e.g.][]{2005ApJ...623L...5F}, as is the span of about one
order of magnitude for the stellar mass and the two orders of
magnitude in lensing mass. With future data, it will be interesting to
compare the enclosed stellar and lensing mass as a function of radius,
going from the star-dominated inner regions to the outer dark
halos. \citet{2011ApJ...740...97L} illustrate this behaviour in their
Fig.~5, but the present sample could go an order of magnitude higher
in mass.

The quick creation of many models for the {\SW} candidates
successfully showed that a subset of citizen scientists are interested
in being involved in more challenging tasks that take some time to
learn. The next step involves recruiting more lensing enthusiasts, as
soon as the next round of {\SW} is started. In the meantime, the
improvements shown in the Appendix will be integrated in the standard
version of {\SpL}. Photometric fitting could also be
integrated into {\SpL}. This would allow experienced citizen
scientists to generate photometric redshifts and stellar masses, and
thus generate preliminary dark-matter maps as soon as a lens-candidate
is identified.



% \section{Acknowledgements}\label{sec:ack}
% We would like to thank the anonymous referee for comments
% that helped to improve this paper.