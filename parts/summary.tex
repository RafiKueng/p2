\section{Summary and conclusions}\label{sec:summary}


We report on a first set of mass distributions and follow up diagnostics for the \SW lens candidates created with a novel approach that aims to be scalable by orders of magnitudes to prepare for the many thousands of lenses the next generation of wide field surveys will yield.

The way of discovering lenses is changing with the introduction of machine learning and citizen science, combined with the coverage of huge areas by modern surveys.
The way mass models are generated however remains the same and isn't prepared for the increasing influx of lenses to be modelled.
We combine the classical approach, where a small team of experts invest many hours into creation of a single model with the approach of citizen science, where a crowd of non trained volunteers make independent contributions.
We recruit motivated citizen scientists and train them to be expert modelers and ask them to create early stage models as soon as a lens candidate is announced.


To assists the expert modelers assessing their models, we introduce a set of diagnostics allowing the categorisation of the lenses, and the 



% way of discovering lenses changed, way of creating models didn't
% factors for c-scientists

Our conclusions are as follows.

Table~\ref{tab:models} is a summary of our results.  It characterises
each modelled system with seven diagnostics, indicating (a)~the image
morphology and how clear or indistinct it is, (b)~whether the mass map
and synthetic lensed image appear to be plausible, and (c)~how the
model mass compares with the estimated stellar and full-halo masses.
Missing entries are due to photometry data not being available, whereas
missing rows are due to models not having been created for this particular
candidate.

Figure~\ref{fig:stelmass} is another summary.  It shows most of the
candidates having stellar and total mass typical of massive
ellipticals.  SW05 is the most massive of all the candidates, with a
mass of galaxy-group scale.  It is a particularly attractive system
for follow-up observations at higher resolution, as it is a large system
with clear multiple-image features, and modelling leaves little doubt
that it is a lens.  The two lowest-mass systems, SW19 and SW42, are
important if they are indeed lenses, because of their low mass.  The
models in both cases are plausible but not entirely satisfactory.

The trend in Figure~\ref{fig:stelmass}, that higher-mass galaxies get
progressively more dark-matter dominated, is expected
, e.g. see \cite{2005ApJ...623L...5F}.
With future data, it would be interesting to compare enclosed
stellar and total mass as a function of radius, going from
star-dominated inner regions to dark halos.
\cite{2011ApJ...740...97L} illustrate this behaviour in their
Figure~5, but the present sample could go an order of magnitude higher
in mass.


% outlook
pop synth not yet automated
