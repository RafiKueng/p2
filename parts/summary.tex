\section{Discussion}\label{sec:summary}

This work follows up on lens candidates reported in
\cite{2016MNRAS.455.1191M} from citizen-science volunteers, with
modelling.  Volunteers role is collaborative.  No robotic competitors.

Table~\ref{tab:models} characterises each modelled system is
(i)~according to the image morphology and how clear or indistinct it
is, (ii)~whether the mass map and synthetic lensed image appear to be
plausible, and (iii)~how the model mass compares with the estimated
stellar and full-halo masses.  Missing rows are due to photometry data
not available.

Figure~\ref{fig:stelmass} summarises the properties of the lens
candidates.  Most appear to have stellar and total mass typical of
massive ellipticals.  SW05 is the most massive of all the candidates,
with a galaxy-group scale mass.  Two systems, SW19 and SW42 have low
mass; the lens models are not entirely satisfactory, but seems
plausible.  If these are lenses, they are very interesting.  In
particular SW42 could be the most dark-matter dominated lens known.

Collorative modelling\footnote{ See ``Collaborative gravitational lens
  modelling\dots'' in {\tt http://letters.zooniverse.org} $\{${\em
    Appears no longer available.}$\}$}
by volunteers.\footnote{In the
  main Space~Warps project for discovering lenses, volunteers in a
  crowd of $\gtrsim10^4$ make independent contributions.  Each person
  is presented with a random selection of survey-patches and invited
  to (in effect) vote on each.  The system estimates each volunteer's
  skill level according to test-patches interspersed with the real
  data, and weights their votes accordingly
  \citep{2016MNRAS.455.1171M}.  There is an active forum for
  volunteers, but since everyone is seeing different data samples with
  minimal overlap, the forum has little if any influence on votes.}

