\section{Summary and conclusions}\label{sec:summary}


We report on a first set of mass distributions and follow up diagnostics for 
the {\SW} lens candidates created with a novel approach that aims to be scalable 
by orders of magnitudes to prepare for the many thousands of lenses the next 
generation of wide field surveys will yield.

The way of discovering lenses is changing with the introduction of
machine learning and citizen science, combined with the coverage of
huge areas by modern surveys.  The way mass models also needs to
change, in order to be prepared for the increasing influx of lenses to
be modelled.  This work is a hybrid of the classical style, where a
small team of experts invest many hours into the creation of a single
model, with the approach of citizen science, where a crowd of amateur
volunteers make independent contributions.  We are a collaboration of
professionals and experienced citizen-science volunteers, aiming to
create early-stage lens models as soon as a lens candidate is found.

To assists volunteer modellers assessing their models, we introduce a
set of diagnostics allowing the qualification of the lenses and the
generated models and present them alongside the generated models. At a
later stage, we encourage modellers to apply those diagnostics
themselves to get a preliminary feedback on the plausibility of their
models.

Table~\ref{tab:models} is a summary of our results.  It characterises
each modelled system with seven diagnostics, indicating (a)~the image
morphology and how clear or indistinct it is, (b)~whether the mass map
and synthetic lensed image appear to be plausible, and (c)~how the
model mass compares with the estimated stellar and full-halo masses.
Missing entries are due to photometry data not being available, whereas
missing rows are due to models not having been created for this particular
candidate.

Figure~\ref{fig:stelmass} is another summary.  It shows most of the
candidates having stellar and total mass typical of massive
ellipticals.  SW05 is one of the most massive of all the candidates, with a
mass of galaxy-group scale.  It is a particularly attractive system
for follow-up observations at higher resolution, as it is a large system
with clear multiple-image features, and modelling leaves little doubt
that it is a lens.
SW04 seems to be even more massive, but the diagnostics leave some doubts
about this model.
The two lowest-mass systems, SW19 and SW42, are
important if they are indeed lenses, as they would be low-mass lenses dominated 
by dark matter.  All the modelled systems have reasonable stellar-mass 
fractions, except for two cases where the stellar-mass fraction is too low 
(halo-matching index $>1$): these are SW42 and SW57.  In the case of SW57, the 
model has poor diagnostics and should be discounted.  The model for SW42, on the 
other hand, is quite convincing --- except for the high halo-matching index.  
If SW42 turns out not to be a lens, that would support the halo-matching index 
as an effective criterion for filtering models.

The trend in Figure~\ref{fig:stelmass}, that higher-mass galaxies get
progressively more dark-matter dominated, is expected,
e.g. see \cite{2005ApJ...623L...5F}, as is the span of about one order
of magnitude for the stellar mass and the two order of magnitudes for the total
mass.
With future data, it would be interesting to compare enclosed
stellar and total mass as a function of radius, going from
star-dominated inner regions to dark halos.
\cite{2011ApJ...740...97L} illustrate this behaviour in their
Figure~5, but the present sample could go an order of magnitude higher
in mass.

The quick creation of many models for the {\SW} candidates successfully showed
that a subset of citizen scientists are interested in being involved in more
challenging tasks that take some time to learn. A next steps involves
recruiting more lensing enthusiasts, as soon as the next round of {\SW} is
started.
In the meantime, the improvements shown in the Appendix have to been integrated
into regular {\SpL} usage.
Additionally, photometric fitting could be integrated into {\SpL}.
This would allow experienced citizen scientists to generate photometric
redshifts and stellar masses, and thus generate preliminary dark-matter maps
as soon as a lens-candidate is identified.


