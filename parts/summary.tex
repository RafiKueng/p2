\section{Discussion}\label{sec:summary}

Table~\ref{tab:models} is a summary of our results.  It characterises
each modelled system with seven diagnostics, indicating (a)~the image
morphology and how clear or indistinct it is, (b)~whether the mass map
and synthetic lensed image appear to be plausible, and (c)~how the
model mass compares with the estimated stellar and full-halo masses.
Missing rows are due to photometry data not available.

Figure~\ref{fig:stelmass} is another summary.  It shows most of the
candidates having stellar and total mass typical of massive
ellipticals.  SW05 is the most massive of all the candidates, with a
mass of galaxy-group scale.  It is a particularly attractive system
for follow-up observations at higher resolution, as it large system
with clear multiple-image features, and modelling leaves little doubt
that it is a lens.  The two lowest-mass systems, SW19 and SW42, are
important if they are indeed lenses, because of their low mass.  The
models in both cases are plausible but not entirely satisfactory.

The trend in Figure~\ref{fig:stelmass}, that higher-mass galaxies get
progressively more dark-matter dominated, is expected
, e.g. see \cite{} %TODO add cite (see, e.g. Ferreras, Saha & Williams 2005)
.  With future data, it would be interesting to compare enclosed
stellar and total mass as a function of radius, going from
star-dominated inner regions to dark halos.
\cite{2011ApJ...740...97L} illustrate this behaviour in their
Figure~5, but the present sample could go an order of magnitude higher
in mass.

