\begin{figure*}
\inclGrid{spl-input}
\caption{Nine of the lens candidates marked up with spaghetti
  diagrams.  Red, blue and green dots are proposed locations for
  maxima, minima and saddle points of the arrival time.  The curves
  help guide the placement of the dots, but their precise appearance
  has no significance.  This selection includes the best-modelled
  systems, but also one case (SW57 at upper right) of unsuccessful
  modelling.  Since the modelling process is collaborative among the
  volunteers, with anyone welcome to contribute new models or modify
  existing ones, there are variant spaghetti diagrams for all the
  modelled systems.  The online supplement displays all the models
  presented for discussion during this work.
\label{fig:splinput}}
\end{figure*}

\begin{figure*}
\inclGrid{kappa_map_interpol}
\caption{Mass distribution $\kappa$ in the systems from
  Figures~\ref{fig:splinput}.
\label{fig:kappa}}
\end{figure*}

\begin{figure*}
\inclGrid{kappa_encl}
\caption{Enclosed mass within a given projected radius, expressed as
  the mean $\kappa$ with a given number of arcsec from the centre of
  the lensing galaxy.  The systems are the same as in
  Figures~\ref{fig:splinput}--\ref{fig:kappa}.  The orange band in
  each panel refers to the full ensemble of mass maps in the model,
  while the red curve is the ensemble average.  The dashed vertical
  line indicates the notional Einstein radius, or where the mean
  enclosed $\kappa$ is unity.  The short vertical arrows indicate the
  positions of the images (maxima, saddle points and minima).
  \label{fig:encl}}
\end{figure*}


\begin{figure*}
\inclGrid{arrival_spaghetti}
\caption{Arrival-time surfaces of the systems from
  Figure~\ref{fig:splinput}--\ref{fig:encl}.
  \label{fig:arriv}}
\end{figure*}

\begin{figure*}
\inclGrid{nsynth}
\caption{Synthetic images of the systems from
  Figures~\ref{fig:splinput}--\ref{fig:arriv}.
  \label{fig:synth}}
\end{figure*}


\begin{figure*}
\includegraphics[width=\linewidth]{img/mlens_vs_mstel_one/mstel_vs_mtot_one}
\caption{Total mass in the model against the estimated stellar mass,
  alongside the values for the whole sample.  The lower-right shaded
  region is unphysical according to the stellar-population models,
  because it gives $M<\Mstel$. The upper-left shaded region is
  unphysical according to abundance matching, because it gives
  $M>\Mhalo$.  That is to say, the unshaded region is
  $0<\haloindex<1$. \label{fig:stelmass}}
\end{figure*}


\begin{table*}
  \caption{Categorisation of SW models}
  \label{tab:models}
  
\begin{tabular}{c c c | c | c c c | c c c}
  \hline
  SWID & ASW id & model id
    & \rot{\shortstack[l]{spaghetti\\type}}
    
    & \multicolumn{1}{|l|}{\rot{\shortstack[l]{blending\\into arc}}}
    & \rot{\shortstack[l]{unseen\\counter\\image}}
    & \rot{\shortstack[l]{lensing\\pertubation}}
    
    & \rot{\shortstack[l]{img reconstr\\reasonable}}
    & \rot{\shortstack[l]{mass reconstr\\reasonable}}
    & \rot{M/L ratio}
  \\ \hline
 
% this is an example entry
%  SW99 & ASW000XXXX & 012345 & 0+0
%    & \NO & \NO & \OK
%    & \OK & \NO & \NO \\

  SW01 & ASW0004dv8 & 
    & X+X
    & \NO & \NO & \NO
    & \NO & \NO & \NO \\
    
  SW02 & ASW000619d & 011489
    & X+X
    & \NO & \NO & \NO
    & \NO & \NO & \NO \\
    
  SW03 & ASW0006mea & 
    & X+X
    & \NO & \NO & \NO
    & \NO & \NO & \NO \\
    
  SW04 & ASW0009cjs & NJ5CC5YJAQ
    & X+X
    & \NO & \NO & \NO
    & \NO & \NO & \NO \\
    
  SW05 & ASW0007k4r & AJIBCHQ6EM
    & 1+2+1
    & \NO & \NO & \NO
    & \OK & \OK & $\sim100$ \\
    
  SW06 & ASW0008swn & BCY2NOUSLK
    & X+X
    & \NO & \NO & \NO
    & \NO & \NO & \NO \\
    
  SW07 & ASW0007e08 & 
    & X+X
    & \NO & \NO & \NO
    & \NO & \NO & \NO \\
    
  SW08 & ASW00099ed & HISGRAIZL2
    & X+X
    & \NO & \NO & \NO
    & \NO & \NO & \NO \\
    
  SW09 & ASW0002asp & 5EKMWWVJHL
    & X+X
    & \NO & \NO & \NO
    & \NO & \NO & \NO \\
    
  SW10 & ASW0002bmc & VQYCYNONVW
    & X+X
    & \NO & \NO & \NO
    & \NO & \NO & \NO \\
    
  SW11 & ASW0002qtn & 3TUJKHGED4
    & X+X
    & \NO & \NO & \NO
    & \NO & \NO & \NO \\
    
  SW12 & ASW0003wsu & 012712
    & X+X
    & \NO & \NO & \NO
    & \NO & \NO & \NO \\
    
  SW13 & ASW00047ae & TGTIIF7HCV
    & X+X
    & \NO & \NO & \NO
    & \NO & \NO & \NO \\
    
  SW14 & ASW0004xjk & 
    & X+X
    & \NO & \NO & \NO
    & \NO & \NO & \NO \\
    
  SW15 & ASW0004nan & QUOGDU2NN6
    & X+X
    & \NO & \NO & \NO
    & \NO & \NO & \NO \\
    
  SW16 & ASW0009bp2 & 013421
    & X+X
    & \NO & \NO & \NO
    & \NO & \NO & \NO \\
    
  SW17 & ASW0005rnb & AAKHMTYTMS
    & X+X
    & \NO & \NO & \NO
    & \NO & \NO & \NO \\
    
  SW18 & ASW0007hu2 & D4UQI6M3ZU
    & X+X
    & \NO & \NO & \NO
    & \NO & \NO & \NO \\
    
  SW19 & ASW0001ld7 & OS3CYAKLRT
    & X+X
    & \NO & \NO & \NO
    & \NO & \NO & \NO \\
    
  SW20 & ASW0002dx7 & 3NYJG67KRT
    & X+X
    & \NO & \NO & \NO
    & \NO & \NO & \NO \\
    
  SW21 & ASW0004m3x & QZROE23AUH
    & X+X
    & \NO & \NO & \NO
    & \NO & \NO & \NO \\
    
  SW22 & ASW0009ab8 & TGM4U2TZBS
    & X+X
    & \NO & \NO & \NO
    & \NO & \NO & \NO \\
    
  SW23 & ASW0003r61 & 002481
    & X+X
    & \NO & \NO & \NO
    & \NO & \NO & \NO \\
    
  SW24 & ASW00050sk & 013406
    & X+X
    & \NO & \NO & \NO
    & \NO & \NO & \NO \\
    
  SW25 & ASW00007mq & 
    & X+X
    & \NO & \NO & \NO
    & \NO & \NO & \NO \\
    
  SW26 & ASW0005ma2 & 5ZZKUM3SWL
    & X+X
    & \NO & \NO & \NO
    & \NO & \NO & \NO \\
    
  SW27 & ASW0006jh5 & 5URN3BQFSV
    & X+X
    & \NO & \NO & \NO
    & \NO & \NO & \NO \\
    
  SW28 & ASW0007xrs & JHC3J2HYV7
    & X+X
    & \NO & \NO & \NO
    & \NO & \NO & \NO \\
    
  SW29 & ASW0008qsm & TOFS7JNGEK
    & X+X
    & \NO & \NO & \NO
    & \NO & \NO & \NO \\
    
  SW30 & ASW0002p8y & 
    & X+X
    & \NO & \NO & \NO
    & \NO & \NO & \NO \\
    
  SW31 & ASW00021r0 & SYTNGELH3Q
    & X+X
    & \NO & \NO & \NO
    & \NO & \NO & \NO \\
    
  SW32 & ASW0004iye & 
    & X+X
    & \NO & \NO & \NO
    & \NO & \NO & \NO \\
    
  SW33 & ASW0003s0m & ECXCIRBDUJ
    & X+X
    & \NO & \NO & \NO
    & \NO & \NO & \NO \\
    
  SW34 & ASW00051ld & 000291
    & X+X
    & \NO & \NO & \NO
    & \NO & \NO & \NO \\
    
  SW35 & ASW0004wgd & VWJ2LNN3VZ
    & X+X
    & \NO & \NO & \NO
    & \NO & \NO & \NO \\
    
  SW36 & ASW000096t & 7IPP7LWVOF
    & X+X
    & \NO & \NO & \NO
    & \NO & \NO & \NO \\
    
  SW37 & ASW00086xq & 
    & X+X
    & \NO & \NO & \NO
    & \NO & \NO & \NO \\
    
  SW38 & ASW0009cp0 & Z6IFI4SLLM
    & X+X
    & \NO & \NO & \NO
    & \NO & \NO & \NO \\
    
  SW39 & ASW0005qiz & 
    & X+X
    & \NO & \NO & \NO
    & \NO & \NO & \NO \\
    
  SW40 & ASW0008wmr & 
    & X+X
    & \NO & \NO & \NO
    & \NO & \NO & \NO \\
    
  SW41 & ASW0008xbu & BFXRMIQEAT
    & X+X
    & \NO & \NO & \NO
    & \NO & \NO & \NO \\
    
  SW42 & ASW00096rm & 4Q3YCEWGLN
    & X+X
    & \NO & \NO & \NO
    & \NO & \NO & \NO \\
    
  SW43 & ASW0001c3j & 5R6UYQZUTI
    & X+X
    & \NO & \NO & \NO
    & \NO & \NO & \NO \\
    
  SW44 & ASW0002k40 & 000899
    & X+X
    & \NO & \NO & \NO
    & \NO & \NO & \NO \\
    
  SW45 & ASW00024id & TSKKYHD3CB
    & X+X
    & \NO & \NO & \NO
    & \NO & \NO & \NO \\
    
  SW46 & ASW00024q6 & 012523
    & X+X
    & \NO & \NO & \NO
    & \NO & \NO & \NO \\
    
  SW47 & ASW0003r6c & 4HC3CREEAD
    & X+X
    & \NO & \NO & \NO
    & \NO & \NO & \NO \\
    
  SW48 & ASW0000g95 & 
    & X+X
    & \NO & \NO & \NO
    & \NO & \NO & \NO \\
    
  SW49 & ASW00007ls & 
    & X+X
    & \NO & \NO & \NO
    & \NO & \NO & \NO \\
    
  SW50 & ASW00008a0 & 
    & X+X
    & \NO & \NO & \NO
    & \NO & \NO & \NO \\
    
  SW51 & ASW0006e0o & 
    & X+X
    & \NO & \NO & \NO
    & \NO & \NO & \NO \\
    
  SW52 & ASW0006a07 & 
    & X+X
    & \NO & \NO & \NO
    & \NO & \NO & \NO \\
    
  SW53 & ASW00070vl & BPAV4GVOPP
    & X+X
    & \NO & \NO & \NO
    & \NO & \NO & \NO \\
    
  SW54 & ASW0007sez & SI4ELBAKL2
    & X+X
    & \NO & \NO & \NO
    & \NO & \NO & \NO \\
    
  SW55 & ASW0007t5y & 
    & X+X
    & \NO & \NO & \NO
    & \NO & \NO & \NO \\
    
  SW56 & ASW0007pga & VHV6RQYYKZ
    & X+X
    & \NO & \NO & \NO
    & \NO & \NO & \NO \\
    
  SW57 & ASW0008pag & 5SXGXQYY6V
    & X+X
    & \NO & \NO & \NO
    & \NO & \NO & \NO \\
    
  SW58 & ASW0007iwp & 4XBJWT3COV
    & X+X
    & \NO & \NO & \NO
    & \NO & \NO & \NO \\
    
  SW59 & ASW00085cp & 
    & X+X
    & \NO & \NO & \NO
    & \NO & \NO & \NO \\
    


  \hline

\end{tabular}

\end{table*}
