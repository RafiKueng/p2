
%%%%%%%%%%%%%%%%%%%%%%%%%%%%%%%%%%%%%%%%%%%%%%%%%%%%%%%%%%%%%%%%%%%%%%%
\section{Developments in SpaghettiLens}

%%%%%%%%%%%%%%%%%%%%%%%%%%%%%%%%%%%%%%%%%%%%%%%%%%%%%%%%%%%%%%%%%%%%%%%
\subsection{Improved synthetic images}\label{subsec:sourcefit}

The mass maps produced by the current implementation of SpaghettiLens
are based on images of point-like features.  No information about
extended images is used, except in so far as they help the user
identify images of point-like features.  The synthetic images offered
to users are rudimentary, corresponding to conical light profiles
(i.e.  circular light profiles with brightness decreasing linearly
with radius).

We have now developed a prototype to improve the generation of
synthetic images, as illustrated in Fig.~\figref{synthimg}. Areas
containing lensed images are selected (green frames in the figure).
The selected areas should be as free as possible from other
sources, including the lensing galaxy. Pixels within the
selected areas are mapped to a grid on the source plane, using bending
angles given by the mass model.  The mapping from lens-plane pixels to
source-plane grid cells is many-to-one, because of image multiplicity
and magnification.  The brightness of each source-plane pixel is set
to the mean of all the lens-plane pixels mapped to it.  Finally, the
mapping is run back to the lens plane.  The result is a synthetic
image.  In effect, one is reconstructing a source-plane brightness map
by least-squares.

The procedure is not yet implemented in SpaghettiLens but can be
applied during post-processing.  The new synthetic images could be used to
improve the mass reconstruction, by weighing the ensemble of maps
according to how good the synthetic images are.

\begin{figure}
  \includegraphics[width=\linewidth]{img/new_synth_img_detailed}
  \caption{Synthetic lensed image with source-profile fitting in SW05
    (J143454.4+522850). Top-left: original image, with areas
    containing lensed images enclosed within green frames.  Top-right:
    synthetic image (coloured arcs) with lensing galaxy and unrelated
    objects in greyscale.  Bottom from left to right: reconstructed
    source in colour, intensity (greyscale), count of lens plane
    pixels per source plane pixel, residual of original image to
    synthetic image.}
  \label{fig:synthimg}
\end{figure}

%%%%%%%%%%%%%%%%%%%%%%%%%%%%%%%%%%%%%%%%%%%%%%%%%%%%%%%%%%%%%%%%%%%%%%%
\subsection{Sub-sampling of the central region}\label{subsec:hires}

The models of simulated lenses in \cite{2015MNRAS.447.2170K} showed a
tendency to produce density profiles which were too shallow, resulting
typically in overestimates of the Einstein radius. Allowing some extra
mass tiles in central region, thus allowing the mass profile to rise
more steeply near the centre, was suggested as a possible cure.
%\textcolor{red}{\bf IF: I don't even know what you mean by the last
%  sentence: were those measures already implemented in the work
%  described in that paper, or were these things considered a solution
%  to the shallow profiles? Please clarify.}


Fig.~\figref{subsampling} shows an experiment with smaller mass
tiles in the inner region.  Replacing the very central mass tile with
9 smaller tiles allows for steeper central profiles.  Doing the same
for the 25 innermost mass tiles allows for even steeper central
profiles, eliminating the systematic shallow profiles.  However, 
it does not provide a completely satisfactory solution, because (a)~it increases
the number of mass tiles by 40\% and significantly increases the
computational time, and (b)~the square boundary between areas with
different tile sizes is rather undesirable.
% 689 / 489 = 1.40899; in practice, runtime is more like x4 - x5!
% runtimes from logfiles: 87/22, 86/21, 85/16
The main modelling work in this paper was, however, done before the
experiments with smaller mass tiles was complete.  Some of the models
presented in this paper apply the intermediate option (corresponding
to the middle panel in Fig.~\figref{subsampling}) while others use
the old system.  We note this paper mainly concerns the enclosed mass
in the outer regions, so shallowness in the central region should not be
an issue.

\begin{figure}
\includegraphics[width=.9\linewidth]{img/hires_comparison/ASW000102p_6941_11_hires_comparison}\\
\includegraphics[width=.9\linewidth]{img/hires_comparison/ASW000102p_6941_13_hires_comparison}\\
\includegraphics[width=.9\linewidth]{img/hires_comparison/ASW000102p_6941_33_hires_comparison}
  \caption{Model improvement resulting from the use of smaller mass tiles
    in the inner region of the mass model.  Shown here are the average
    enclosed $\kappa$ within a given projected radius, for three
    different reconstructions of a simulated lens (sim) from
    Space~Warps.  In each panel, the dashed blue curve is the correct
    answer.  The orange band represents the statistical ensemble from
    SpaghettiLens, the orange line shows the ensemble mean.  Locations of
    images (maximum, saddle point, minimum) are marked with vertical
    arrows.  The radial value at $\kappa=1$ is the effective
    Einstein radius, \ER. The upper panel is taken from
    \citet{2015MNRAS.447.2170K}, see Fig.~3 of that paper.  The middle
    panel is the result when the innermost mass tile is replaced by 9
    smaller tiles.  The lower panel results from replacing the
    innermost 5$\times$5 tiles with 9 smaller tiles each.}
  \label{fig:subsampling}
\end{figure}

%%%%%%%%%%%%%%%%%%%%%%%%%%%%%%%%%%%%%%%%%%%%%%%%%%%%%%%%%%%%%%%%%%%%%%%
\subsection{Parametrization of pixel models} \label{subsec:parameter}

In order to fit the set of pixelated models to a single parametrized
model, a programme was written that took a parametrized function and
subtracted from it the mean and the principal components (PCs) of the
data. This created the residual function.  The number of principal
components in the analysis was varied, to test how this affected the
output. It was found that 5 PCs gave a reasonable approximation. A
masking function was added, selecting only the data points that fell
inside the image of the lens, and the PCs were clipped in order to
keep the values inside the region of the ensemble. Any value higher
than the clip was set to be the clip value. This was chosen to 2.5
because, assuming that the data follows a Gaussian error distribution,
almost all the values for the variance should lie between 2 and 3
standard deviations from the mean. Minimising the residuals function
produces the set of parameters that fit the parametrized function to
the original pixelated ensemble most closely.  A least squares fit was
used to perform this minimisation.  The parametrized model function
was obtained from the gravitational potential of an isothermal
ellipsoid mass distribution \citep{2001astro.ph..2341K}.  This model
is frequently used to describe gravitational lenses, giving good fits
to the observations.  The isothermal ellipsoid model outputs three
useful parameters: the radius of the Einstein ring, the ellipticity of
the model and the angle of the ellipticity from the vertical, giving
the orientation of the galaxy.  By applying this model to simulated
lenses for which the values of these parameters were already known, it
was possible to assess the accuracy of the methodology, before
applying the model to the candidate lensing galaxies.  Preliminary
results on the recovery of Einstein radii are shown in
Fig.~\ref{fig:parameter}.

\begin{figure}
  \includegraphics[width=\linewidth]{img/rE_comp/rE_comp.png}
  \caption{ Comparison of Einstein radii \ER\ obtained from mass tiles
    to the results from a parameterized model to test the performance
    of the algorithm.  The parameterized model was generated using
    principal component analysis on the ensemble of models.  The blue
    dashed line represents a perfect recovery of \ER.  }
  \label{fig:parameter}
\end{figure}



