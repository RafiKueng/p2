
%%%%%%%%%%%%%%%%%%%%%%%%%%%%%%%%%%%%%%%%%%%%%%%%%%%%%%%%%%%%%%%%%%%%%%%
\section{Stellar and halo mass estimates}
\label{sec:stellar-mass}

The stellar masses of the lens galaxies are derived by comparison of
the photometric data with stellar M/L estimates from population
synthesis models.  In principle, a detailed analysis of the spectral
energy distribution is needed to derive accurate stellar masses
\citep[e.g.][]{2009ApJS..185..253G,2011MNRAS.418.1587T}.
Fortunately, however, the 4000\thinspace\AA~break in stellar spectra
makes things considerably easier, as this spectral feature is (a)~very
sensitive to the age of a stellar population, and (b)~possible to
constrain even with just two broad-band flux measurements on either
side of it.  As a result,
estimates to within $\sim$0.3\,dex in $\log(\Mstel/\Msun)$ can be derived
with a single colour, preferably tracing a rest-frame colour similar
to $U-V$ \citep[see Fig.~1 of][]{2008MNRAS.383..857F}, assuming
a universal initial mass function (IMF). There is
evidence from detailed absorption line strength analysis that massive
galaxies can feature a non-standard IMF 
\citep[e.g.][]{2013MNRAS.429L..15F}. However, these variations --
towards a bottom-heavy distribution -- are typically found in the
cores of massive early-type galaxies \citep{2016MNRAS.457.1468L}. The
effect of these variations on the stellar M/L of lensing systems is
still rather controversial
\citep{2015MNRAS.449.3441S,2016MNRAS.459.3677L}.

In this paper we further simplify the analysis by assuming a
relationship between the apparent total magnitude and stellar mass, at
the redshift of the lens.  For typical stellar-population parameters,
the variation of this relation is at most $\Delta\log$M$_s\sim$1\,dex.
A further potential systematic can arise from contamination of the
light of the lensing galaxy by the lensed background source.  Reducing
or eliminating the latter would require detailed fitting of light
distributions for each candidate \citep[see][]{2011ApJ...740...97L},
which we have not yet attempted.  Nonetheless, because the lensing
masses range over two orders of magnitude, it is still interesting to
compare them with rough estimates of stellar mass.

We make use of the \citet{2003MNRAS.344.1000B} models to derive two
functional forms of the stellar mass with respect to the $i^{\prime}$-band
magnitudes. The models have solar metallicity, with a Chabrier IMF,
and assume two different age trends: a ``young'' model, with a
constant 500\,Myr age at all redshifts, and an ``old'' model where the
age is the oldest one possible at each redshift, adopting a standard
$\Lambda$CDM model with $H_0=70$\,km\,s$^{-1}$\,Mpc$^{-1}$ and
$\Omega_m=0.3$.

Fig.~\ref{fig:stelmass} shows a comparison of stellar and lensing mass
in our sample.  The comparatively large span of the error bars in
stellar mass (horizontal axis) shows the range between the masses
derived using the two age trends respectively, and lies between 0.4 and
0.8\,dex.  It will be improved in future work by the use of available
optical and NIR magnitudes to derive more accurate constraints on the
stellar populations.  In addition, we also derive halo masses for the
lenses using an abundance-matching formula.  This technique matches
the distribution function of observed stellar mass in galaxies with
that of dark-halo masses from $N$-body simulations, to define  a simple
relation between stellar mass and halo mass.  We emphasize that a halo
mass from abundance matching should be considered an ``average''
estimate, and a significant scatter can be expected as galaxies with
the same stellar mass can be found in different environments. We refer
the reader to \cite{2012MNRAS.424..104L} for an assessment of the
effect of abundance matching on the derivation of dark matter halo
properties in lensing galaxies. We follow the prescription of
\citet{2010ApJ...710..903M}, namely:
\begin{equation}
    \frac{\Mstel}{\Mhalo} = \frac{2C_0}{(\Mhalo/M_1)^{-\beta} +
                                     (\Mhalo/M_1)^\gamma}
\end{equation}
\begin{align*}
    C_0 &= 0.02820, & M_1 &= 10^{11.884} M_\odot \\
    \beta &= 1.057, & \gamma &= 0.556.
\end{align*}
Fig.~\ref{fig:stelmass} may be compared with Fig.~4 in
\cite{2011ApJ...734...69M}.
The comparison of lensing and stellar mass produces the last 
of our model diagnostics, defined as a halo-matching index:
\begin{equation}
\haloindex \equiv \frac{\ln(M/\Mstel)}{\ln(\Mhalo/\Mstel)}
\end{equation}
that relates the observed lensing to stellar mass, with the
global ratio expected if the host halo corresponds to the
average value derived by abundance matching. Several cases
for $\haloindex$ can be considered:
\begin{itemize}
\item $\haloindex < 0$ is unphysical because $M<\Mstel$.
\item $\haloindex = 0$ means the stellar mass exactly accounts for the
  lensing mass (i.e. no dark matter affects the lensing model).
\item $0 < \haloindex < 1$ is the typical situation, where the lens
  includes stars and dark matter, but not the full halo.
\item $\haloindex = 1$ means that the lens consists of the entire halo.
\item $\haloindex > 1$ is in tension with abundance-matching, because the
  lensing mass exceeds the expected halo mass.
\end{itemize}
The halo-matching index expresses whether the lensing mass is
plausible given the flux received from the candidate lensing galaxy.

Fig.~\ref{fig:stelmass} and Table~\ref{tab:models} show that most of the
candidates have stellar
and lensing masses typical of massive ellipticals\footnote{The mass values
themselves are given in the online version of Table~\ref{tab:models}}.
\sw{05} is one of the most massive of all the candidates, corresponding
to a galaxy-group mass scale.  It is a particularly attractive system
for follow-up observations at higher resolution, as it is a large
system with clear multiple-image features. Modelling leaves little
doubt that it is a lens.  \sw{04} seems to be even more massive, but the
diagnostics leave some doubts about the validity of this model.  The
two lowest-mass systems, \sw{19} and \sw{42}, are important if they are
indeed lenses, as they would be low-mass lenses dominated by dark
matter.  All the modelled systems have reasonable stellar-mass
fractions, except for two cases where the stellar-mass fraction is too
low ($\haloindex > 1$): these are \sw{42} and \sw{57}.  In the case of \sw{57},
the model has poor diagnostics and should be discarded.  The model
for \sw{42}, on the other hand, is quite convincing -- except for the
high halo-matching index.  If \sw{42} turned out not to be a lens, that
would support the halo-matching index as an effective criterion to
discriminate models.

