\section{Image morphology}\label{sec:morph}

A modeller's main input to SpaghettiLens is a sketch of arrival-time
contours on a lensed system.  It gives the locations (or best-guess
locations) of images of point-like sources, along with their ordering
in arrival time, and their parities (minumum, saddle-point or
maximum).  Such a sketch, which we call a spaghetti diagram, tends to
resemble the form of lensed arcs, but it also in effect 
encodes a proposal for its mass distribution.

Figure~\ref{fig:splinput} shows cutouts of the Space~Warps image,
marked up with a spaghetti diagram.

We show modelling results for nine of the lens candidates.  The first
five (see Figures~\figref{SW05}--\figref{SW29}) are the most
convincingly modelled systems.  The next three cases
(Figures~\figref{SW42}--\figref{SW19}) are less good but still very
plausible models, and are representative of the majority of the
sample.  The last two are examples where the models were unconvincing
(Figures~\figref{SW36}) or completely failed (\figref{SW57}).


Let us first consider from SW05 (J143454.4+522850).
In SW05 there are four distinct lensed images, and the spaghetti
diagram proposes that they are a minimum nearest to the lensing
galaxy, a close minimum-saddle pair, and a minimum further away. In
Table~\ref{tab:models} we refer to such configurations as IQ for
inclined quad.

Figure~\figref{SW42} shows SW42.  The image morphology is similar to
SW05 in Figure~\figref{SW05}, but the lens is much smaller on the sky.

Proceeding to Figure~\figref{SW28}, we have one image close to the
galaxy and an arc further away, which is interpreted as a blend of
three images (a saddle point with two minima on either side).  We call
this a long-axis quad or LQ configuration.  It is an indication of a
mass distribution elongated along the arc-counterimage direction
(along EW in this case).

In Figures~\figref{SW02}--\figref{SW29} we see three candidates with
an arc close the lensing galaxy and one image further away.  The arc
is interpreted three images (a minimum with two saddle points on
either side).  We call this a short-axis quad or SQ configuration.  It
is an indication of a mass distribution elongated perpendicular to the
arc-counterimage direction.

