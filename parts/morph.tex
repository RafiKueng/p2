\section{Image morphology}\label{sec:morph}

A modeller's main input to the lens-modelling process is a markup of
the candidate lens system, which we call a spaghetti diagram.
This is a sketch of the arrival-time surface from a point-like source,
with proposed locations of maxima, minima and saddle points, and an
implied time-ordering of the images.  Such information encodes a
starting proposal for the mass distribution.  A spaghetti diagram is
thus a completely abstract construction, and moreover it refers to a
simplified system with a point source.  However, spaghetti diagrams
are intuitive because they tend to resemble the form of lensed arcs,
and of course they are simple to draw, and easy to vary and refine in
an open collaborative environment.  This makes them very practical for
non-professional lens enthusiasts in the citizen-science community.
Details and tests are given in \cite{2015MNRAS.447.2170K}.

We now discuss the diagnostics that can be taken from the process of
drawing the spaghetti diagram, even before detailed mass-modelling
takes place.  Figure~\ref{fig:splinput} shows nine examples, each
consisting of a cutout of the Space~Warps image of a lens candidate,
marked up with a spaghetti diagram.

All the examples in Figure~\ref{fig:splinput} identify five locations:
the centre of the main lensing galaxy, which is also a maximum of the
arrival time surface (red dots); two minima (blue dots); and two
saddle points (green dots and also self-intersections of the curves).
Only two (SW05 and SW42) of the nine systems, however, have five
distinct features in plausible locations.  In the other seven
examples, an arc is interpreted as a blend of three or four images.
This gives the `unblended images' diagnostic.  Note that this
characterisation could be different if the spaghetti diagram is
different.  For example, SW28 has also been modelled with the arc on
the right interpreted as a single image, rather than as three images
as shown in the figure; for such a model, `unblended images'
diagnostic would be true.

The second diagnostic is whether all images are visible.  For example,
we see in SW58 at the top left of Figure~\ref{fig:splinput} that an
image at the second minimum is conjectural and does not correspond to
any visible feature.

The third diagnostic is whether the lens is fairly isolated, or
whether there are other galaxies in the field that probably contribute
significantly to the lensing.  For this, we do not consider stars or
other clearly foregound objects.  Additional galaxies contributing to
the lensing mass can be
marked by the volunteer alongside the spaghetti diagrams.  An example
can be seen as the grey dot and circle in the cutout with SW57 at the
top right of Figure~\ref{fig:splinput}.  Objects marked in this way
are modelled as point masses.  Other possible contributors to lensing
are galaxies or groups that are not in the immediate vicinity of
the lensed images, yet massive enough to have an effect.  These are
accounted for by allowing a constant but adjustable external shear.

We remark that blended images or missing images or perturbing galaxies
do not imply that a given candidate is unlikely to be a lens.  They
mean, rather, that the models are more uncertain and perhaps could be
more easily improved by trying further variations in the markup.

The fourth diagnostic is based on the fact that the arrangement of
lensed images of a pointlike source through a non-circular
gravitational lens depends on the location of the source relative to
the long and short axes of the lens.  This dependence is quite robust
and independent of many other details of the lensing mass
distribution.  Sources close to being dead-centre behind a lens tend
to produce quads, sources at larger transverse distance tend to
produce doubles.  We denote these as Q and D respectively, and add a
prefix to the Q systems, as follows: we write LQ if the source is
inferred as displaced along the long axis of the lens, SQ if displaced
along the short axis, IQ if inclined to both axes, CQ if only very little to
no displacement.  Although the
unlensed source and its location are obviously not seen, the LQ, SQ, IQ and CQ
cases correspond to easily-seen image morphologies \citep[see,
  e.g.,][]{2003AJ....125.2769S}.
\begin{itemize}
\item The simplest is the SQ case: this creates a saddle point and two
  minima in an arc, with the second saddle point on the other side of
  the lensing galaxy, and closer to the galaxy than the arc is.  SW58
  and SW28 in the upper row of Figure~\ref{fig:splinput} are typical
  examples of SQ.  If the source would move outwards along the short
  axis, the minimum-saddle-minimum would merge into a single minimum,
  leaving a D system; the transition is known as cusp catastrophe.
\item The middle row of Figure~\ref{fig:splinput} shows three IQ
  systems, SW05, SW42 and SW19.  This type has a characteristic
  asymmetry, often with two images close together.  If the source
  would move outwards, the minimum-saddle pair would merge and
  mutually cancel, again leaving a D system; the transition is known
  as a fold catastrophe.
\item The lower row of Figure~\ref{fig:splinput} shows three LQ
  systems, SW09, SW29 and SW02, and the failed model SW57 at top right
  also possibly belonging to this category.  Here the images have a
  fairly symmetric arrangement with an arc and a counter-image on the
  other side, but the spaghetti diagram is completely asymmetric.  If
  the source would move outwards along the long axis, the
  saddle-minimum-saddle would merge into a single saddle --- another
  form of cusp catastrophe.  LQ can be visually distinguished from SQ
  from the relative distances of the arc and the counter-image.  For
  LQ the arc is closer, for SQ the counter-image is closer.
\end{itemize}
CQ systems are often called `cross' or `Einstein-cross' systems; IQ
systems are sometimes called `folds', with `cusp' commonly used for
both SQ and LQ.  The labels `short-axis quad' and so on are not
standard in the literature, but the morphological classification they
express is familiar to experienced modellers.  Hence they can be
useful to researchers wishing to apply other modelling methods to the
same systems.

